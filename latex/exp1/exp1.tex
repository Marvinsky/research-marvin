
%\documentclass[11pt,a4paper,oneside]{report}
\documentclass[a4paper,12pt]{article}
\synctex=1

\usepackage{pslatex,palatino,avant,graphicx,color}
\usepackage[margin=2cm]{geometry}

\usepackage{mathptmx}       % selects Times Roman as basic font

\usepackage{helvet}         % selects Helvetica as sans-serif font
\usepackage{courier}        % selects Courier as typewriter font
\usepackage{type1cm}        % activate if the above 3 fonts are
                            % not available on your system

\usepackage{makeidx}         % allows index generation
\usepackage{graphicx}        % standard LaTeX graphics tool
                             % when including figure files
\usepackage{multicol}        % used for the two-column index
\usepackage{multirow}
\usepackage{booktabs}

\usepackage[bottom]{footmisc}% places footnotes at page bottom

\usepackage[utf8]{inputenc}
\inputencoding{utf8}

\usepackage[square,sort]{natbib}
\bibliographystyle{unsrtnat}


\usepackage[utf8]{inputenc}
\usepackage[english]{babel}

\usepackage{longtable}
\usepackage{pdflscape}

%for formula references
\usepackage{amsmath}

\newtheorem{mydef}{Definition}

\usepackage{changes}
%triangledown
\usepackage{latexsym}
\usepackage{amssymb}
\usepackage{amsfonts}
\usepackage{amsthm}

\usepackage[]{algorithm2e}

%langle
\usepackage{scalerel}
\usepackage{graphicx}


%width of the column
\usepackage{array}
\newcolumntype{L}[1]{>{\raggedright\let\newline\\\arraybackslash\hspace{0pt}}m{#1}}
\newcolumntype{C}[1]{>{\centering\let\newline\\\arraybackslash\hspace{0pt}}m{#1}}
\newcolumntype{R}[1]{>{\raggedleft\let\newline\\\arraybackslash\hspace{0pt}}m{#1}}

%rotate the name of the column
\usepackage{adjustbox}
\usepackage{array}

\newcolumntype{R}[2]{%
    >{\adjustbox{angle=#1,lap=\width-(#2)}\bgroup}%
    l%
    <{\egroup}%
}
\newcommand*\rot{\multicolumn{1}{R{90}{1em}}}% no optional argument here, please!

%checkmark
\usepackage{tikz}
\def\checkmark{\tikz\fill[scale=0.4](0,.35) -- (.25,0) -- (1,.7) -- (.25,.15) -- cycle;} 


\makeindex

\begin{document}


\begin{titlepage}

\begin{center}
\vspace*{-1in}
\begin{figure}[htb]
\begin{center}
\includegraphics[width=8cm]{./image/ufv1}
\end{center}
\end{figure}

CENTRO DE CIENCIAS EXATAS E TECNOLOGICAS - CCE\\
\vspace*{0.15in}
DEPARTAMENTO DE INFORMATICA \\
\vspace*{0.6in}
\begin{large}
REPORT:\\
\end{large}
\vspace*{0.2in}
\begin{Large}
\textbf{COMPARISON BETWEEN SS AND IDA*} \\
\end{Large}
\vspace*{0.3in}
\begin{large}
%A Thesis Project submitted by Marvin Abisrror for the degree of Master to the PPG\\
\textbf{Marvin Abisrror Zarate} \\
MSc Student in Computer Science \\
\end{large}
\vspace*{0.3in}
\rule{80mm}{0.1mm}\\
\vspace*{0.1in}
\begin{large}
Levi Henrique Santana de Lelis \\
(Advisor)
\

\

\

\
Santiago Franco \\
(Co-Advisor)
\

\

\

\

\
\end{large}
VIÇOSA - MINAS GERAIS\\
JULY - 2015
\end{center}
\end{titlepage}


\tableofcontents
\newpage

\section{Introduction}
The new approach for selecting a subset of heuristics functions for domain-independent planning has two objectives: First, make a selection of heuristics from a large set of heuristics with the goal of reducing the running time of a search algorithm employing the subset functions. Second, find out if the prediction of Stratified Sampling (SS) might be helpful in selecting a subset of heuristics to guide the A* search.

In order to achieve the first objective we present The Greedy Algorithm, which provides a good approximation to the optimal solution of the NP-hard optimization problem \citep{krause2012submodular}. 

In order to achieve the second objective we use the \textit{relative usigned error} to probe the accuracy of the predictions of SS with respect to IDA*. We know that SS does not make even reasonable predictions for the number of nodes expanded by A*. Nevertheless, even though SS produces poor predictions for the number of nodes expanded by A*, we would like to verify whether these predictions can be helpful in selecting a subset of heuristics to guide the A* search.

This report const of three sections, the first section is the introduction, the second is the experiment 1 which contains 4 tables showing the results of the \textit{relative usigned error}. and the last section in the conclusion.

\section{Comparison of SS with IDA*}
  


\newpage
%Imports the bibliography file "references.bib"
\bibliography{references}
%\bibliographystyle{references}

\end{document}
