%Empieza configuracion de capitulo
\setstretch{1.0}
\titleformat{\chapter}[block]{\Large\bfseries}{CAP'ITULO \Huge\thechapter\vspace{25 pt}}{0 pt}{\\\fontsize{26}{36}\selectfont}
\titlespacing{\chapter}{0 pt}{30 pt}{50 pt}[0 pt]
\titleformat{\section}{\Large\bfseries}{\thesection}{0 pt}{\hspace{30 pt}}
\titleformat{\subsection}{\large\bfseries}{\thesubsection}{0 pt}{\hspace{30 pt}}
\pagestyle{fancy}
\fancyhead[LO,LE]{\footnotesize\textit{\leftmark}}
\fancyhead[RO,RE]{\thepage}
\fancyfoot[CO,CE]{}
%Termina configuracion de capitulo

\chapter{Marco Te'orico} %Cambia Marco Te'orico al nombre de tu capitulo
\setstretch{1.5} %Regresa el interlineado a 1.5

\normalsize

\section{Estado del arte}
\noindent
En este cap'itulo se describe el estado del arte con un pre'ambulo de los primeros sistemas de posicionamiento y luego se hace referencia a la aplicaci'on de t'ecnicas de miner'ia de datos basados en datos de los diferentes sistemas de posicionamiento. Tambi'en se mencionan t'ecnicas de miner'ia de datos aplicadas a optimizar el almacenamiento de informaci'on en tiempo real, y varias aplicaciones del uso intensivo de datos extra'idos de los sat'elites de posicionamiento.\\

Adem'as se nombran varios de los centros de investigaci'on que actualmente est'an trabajando o han trabajado a partir de los datos de los sistemas de georreferenciaci'on, como tambi'en los diversos modelos de negocio que surgen de estos datos.

\subsection{Los primeros sistemas de posicionamiento}
\noindent
El antecedente inmediato de los sistemas de posicionamiento modernos de hoy en d'ia es el sistema de navegaci'on satelital marina (NNSS), tambi'en llamado Sistema de Tr'ansito o \emph{Transit System}. Este sistema se concibi'o a finales de 1950 y se desarroll'o en la d'ecada de 1960 por el servicio militar de EE.UU., sobre todo, para determinar las coordenadas y tiempo de los buques en el mar y para uso militar en tierra. El uso civil de este sistema de sat'elites fue finalmente liberado a finales de 1964 para su manipulaci'on; el sistema se utiliza en todo el mundo tanto para la navegaci'on como la topograf'ia.\\

La primera oferta razonable en el uso de sat'elites para la navegaci'on naci'o durante la investigaci'on de la posible aplicaci'on de las tecnolog'ias de radio - astronom'ia para aeronavegaci'on dirigido por el Prof. V.S. Shebshaevich, en la Academia Militar de Ingenier'ia de Leningrado Mozhaiskii en 1957.\\

Las investigaciones adicionales para aumentar la precisi'on de las definiciones de navegaci'on, el apoyo mundial, la aplicaci'on diaria y la independencia de las condiciones meteorol'ogicas, permitieron el desarrollo del sistema \emph{Tsikada} ruso (o tambi'en conocido como Cicada) que transmite las mismas dos frecuencias portadoras como Tr'ansito, y es similar a la misma con respecto a las precisiones alcanzables. Diez sat'elites de 'orbita baja se desplegaron en dos constelaciones complementarias: uno militar compuesto por una constelaci'on de seis sat'elites, y una red civil compuesta por una constelaci'on de cuatro sat'elites. Al contrario de Tr'ansito, el sistema \emph{Tsikada} sigue funcionando.

\subsection{Los sistemas de posicionamiento: actuales y futuros }
\noindent
El sistema de navegaci'on con tiempo y rango (NAVSTAR), y el Sistema de Posicionamiento Global (GPS) fue desarrollado por el 'area militar de EE.UU. para superar las deficiencias de los sistemas anteriores.\\

El Sistema Global de Navegaci'on por Sat'elite (GLONASS) es el equivalente ruso del GPS y es operado por el ej'ercito ruso. GLONASS difiere de GPS en t'erminos del segmento de control, el segmento de espacio, y la estructura de la se'nal.\\

Galileo es la contribuci'on europea al futuro de los GNSS (Global Navigation Satellite System). Un sistema chino llamado \emph{Compass}, que es la evoluci'on del sistema regional de primera generaci'on \emph{Beidou}, se encuentra actualmente en fase de desarrollo.\\

Como se mencion'o, GNSS \cite{hofmann-wellenhof_gnss--global_2008} implica varios sistemas existentes, como GPS, GLONASS o Galileo. Adem'as, estos sistemas se complementan con los sistemas de aumentaci'on basados en el espacio (\emph{space-based augmentation systems - SBAS}) o sistemas de aumentaci'on basado en tierra (\emph{ground-based augmentation systems - GBAS}). Ejemplos de \emph{SBAS} son el sistema de los EE.UU. de 'area amplia de aumento (\emph{wide-area augmentation system - WAAS}), la 'orbita de servicio europeo de superposici'on de navegaci'on (\emph{European geostationary navigation overlay service - EGNOS}) o el sat'elite de transporte multifuncional japon'es (\emph{multifunctional transport satellite - MTSAT}) basado en el espacio del sistema de aumentaci'on (\emph{space-based augmentation system - MSAS}). Estos sistemas aumentan la 'orbita media (\emph{medium earth orbit - MEO}) con constelaciones de sat'elites geoestacionarios.

\subsection{Glonass}
\noindent
GLONASS es la abreviatura del ruso "\emph{Global'naya Navigatsionnaya Sputnikovaya Sistema}", traducida a su equivalente del espa'nol, esto significa Sistema Global de Navegaci'on por Sat'elite o en ingl'es, \emph{Global Navigation Satellite System}.\\

A mediados de 1970, la ex Uni'on de Rep'ublicas Socialistas Sovi'eticas (URSS) inici'o el desarrollo de GLONASS en base a las experiencias con el sistema de sat'elites \emph{Doppler Tsikada}. La \emph{Academician M.F. Reshetnev's State Unitary Enterprise of Applied Mechanics} ha sido la principal contratista responsable de la elaboraci'on y la aplicaci'on general del sistema.\\

Los subcontratistas son el Instituto Ruso de Investigaci'on Cient'ifica de la Industria Espacial (\emph{Russian Scientific-Research Institute of Space Industry}) y el Instituto Ruso de Radionavegaci'on y Hora (\emph{Russian Institute of Radionavigation and Time}). Estos institutos son los responsables de la vigilancia y el control, y tambi'en participaron para un desarrollo adecuado de los receptores y relojes.\\

Seg'un \emph{Coordination Scientific Information Center}, y como se define en el documento de control de interfaz GLONASS, el prop'osito de GLONASS es proporcionar un n'umero ilimitado de usuarios en el aire, mar, y cualquier otro tipo de usuarios en cualquier tiempo un posicionamiento tridimensional, de medici'on de velocidad y de tiempo en cualquier parte el mundo o en el espacio cercano a la Tierra.\\

Al ser GLONASS operado por las fuerzas militares rusas, casi no hubo informaci'on detallada cuando se dio a conocer. M'as tarde, este d'eficit de informaci'on cambi'o. En 1988, en una reuni'on de la Comisi'on Especial sobre el Futuro de Sistemas de Navegaci'on A'erea (\emph{Special Committee on Future Air Navigation Systems}) de la Organizaci'on de Aviaci'on Civil Internacional (\emph{International Civil Aviation Organization - OACI}), se present'o un documento con detalles t'ecnicos de GLONASS y la URSS ofreci'o el uso gratuito de este sistema satelital. M'as adelante, en marzo de 1995, el Gobierno de la Federaci'on de Rusia lanz'o el Decreto n'umero 237, donde el Ministerio de Defensa de la Federaci'on Rusa, la Agencia Espacial Federal Rusa y el Ministerio de Transporte de la Federaci'on de Rusia se comprometieron a proporcionar el despliegue del sistema de navegaci'on global por sat'elite GLONASS, y el inicio de su operaci'on con su asignaci'on completa en 1995 con el fin de dar servicio a los usuarios civiles, militares nacionales y los usuarios civiles extranjeros de acuerdo a los compromisos existentes.\\

Las pruebas de vuelo del sistema de alta altitud de navegaci'on por sat'elite, GLONASS, se iniciaron en octubre de 1982 con el lanzamiento del \emph{Kosmos-1413}. El sistema GLONASS se puso en las pruebas de funcionamiento en 1993. Para 1995 se form'o la 'orbita completa compuesta por 24 sat'elites. El sistema proporciona navegaci'on global continua de todos los tipos de usuarios con diferentes niveles de requisitos de calidad para apoyo a la navegaci'on. La reducci'on de la financiaci'on de la industria espacial en 1990 condujo a la degradaci'on de la constelaci'on GLONASS. A'nos m'as tarde el Presidente y el Gobierno ruso aprobaron una serie de documentos de pol'itica, incluyendo el programa federal ``Sistema Global de Navegaci'on'' para proporcionar los sistemas de seguridad y del progreso. Apoy'o la creaci'on de un campo de navegaci'on global para determinar las coordenadas de los objetos con un alto grado de exactitud y fiabilidad, la introducci'on de tecnolog'ias de navegaci'on por sat'elite en la gesti'on del tr'afico de informaci'on, la mejora de la seguridad en el sector del transporte por carretera del pa'is, una importante reducci'on de los costos de operaci'on.

\subsection{Sat'elites y miner'ia de datos}
\noindent
Actualmente, la enorme cantidad de informaci'on almacenada en las organizaciones de cualquier sector del mercado es una fuente potencial de conocimiento para ser explorado y extra'ido. Igualmente los sat'elites de posicionamiento se han vuelto una gran fuente de informaci'on para diversas empresas, especialmente, aquellas que trabajan o investigan a partir de datos geo-espaciales.\\

Es en este punto donde diferentes procesos, t'ecnicas, metodolog'ias y 'areas del conocimiento se unen para sacar provecho y dar soluciones a los grandes retos que implica la utilizaci'on de datos de posicionamiento, su infraestructura, su extracci'on, su procesamiento y su aplicaci'on.\\

La extracci'on de conocimiento a partir de las fuentes existentes de informaci'on es un 'area de desarrollo clave para desbloquear las relaciones desconocidas entre los diferentes puntos de datos. La miner'ia de datos es una t'ecnica que utiliza m'etodos de inteligencia artificial para extraer relaciones previamente desconocidas. Se convierte en un factor cuando se deben analizar grandes vol'umenes de datos, como en el caso de las agrupaciones de sat'elites.\\

Recientemente son m'as las empresas, organizaciones, centros de investigaci'on y agencias que est'an recurriendo a la utilizaci'on de t'ecnicas de miner'ia de datos. Lo hacen por varias razones, entre ellas la necesidad de responder a las cambiantes necesidades de los clientes, y del mercado. Las mismas herramientas que utilizan las empresas tambi'en se pueden aplicar a la tecnolog'ia usada en los  sat'elites.\\

Por ejemplo, los registros en las empresas contienen informaci'on hist'orica acerca de la misma empresa, los clientes y las transacciones, y por lo general se asocian a empresas, como los bancos, las grandes tiendas por departamentos, tarjetas de cr'edito, seguros, telecomunicaciones, salud y agencias gubernamentales. Las empresas de este tipo acumulan decenas de miles de registros cada d'ia. Estos registros est'an en la forma de los datos operativos tales como los puntos de venta, entrada de pedidos, catalogaci'on, entre otros.\\

De manera similar los sat'elites tambi'en acumulan registros, pero en la forma de puntos de datos de telemetr'ia. Varios cientos de miles de puntos de datos tienen el formato de tramas de telemetr'ia en cada paso de tiempo. Estos datos se almacenan en bases de datos para ser analizados. Los puntos de datos son la base de los datos hist'oricos que se acumulan y pueden ser extra'idos para relacionar la informaci'on.\\

Una evidencia de esto, es el an'alisis de los datos de telemetr'ia\cite{self_use_2000} por sat'elite a trav'es de la miner'ia de datos, lo cual provee en gran parte las mismas ventajas que son frecuentes en la comunidad empresarial. Aqu'i, la identificaci'on y categorizaci'on de los par'ametros se llev'o a cabo dentro del almac'en de datos (\emph{Data Warehouse}) donde los datos son preparados (normalizados y reformateados). Una vez procesados estos datos son pasados a un \emph{Data Mark}, donde los expertos y los usuarios pueden encontrar la informaci'on de inter'es ubicada en categor'ias.\\

La miner'ia de datos es el 'area que mayores aportes ha realizado para el uso intensivo de los datos y el descubrimiento de nuevo conocimiento, aunque tambi'en ha sido utilizado para la optimizaci'on de los recursos de procesamiento y almacenamiento, como se evidencia a trav'es de un sistema de distribuci'on de im'agenes por sat'elite en l'inea \cite{azevedo_application_2007}, en el cual se extrajo el conocimiento sobre el uso de las im'agenes con el objetivo de detectar el uso potencial de una imagen real de la base de datos del sat'elite. Los resultados obtenidos indicaron que el uso de t'ecnicas de miner'ia de datos puede ayudar en la automatizaci'on de la elecci'on de las im'agenes que se procesan y se almacenan antes. Como consecuencia, se puede representar una mejora en el servicio al cliente y puede conducir a un mejor uso del espacio de almacenamiento y los recursos de procesamiento.\\

\subsection{Aplicaciones con datos satelitales}
\noindent
Los datos telem'etricos son la 'unica fuente para identificar y predecir las anomal'ias en los sat'elites artificiales. Existen personas especializadas en el an'alisis de estos datos en tiempo real, pero su gran volumen hace que este an'alisis sea extremadamente dif'icil. Por tal motivo, se aplic'o la t'ecnica de algoritmos de agrupamiento para ayudar a los operadores y analistas a realizar la tarea de an'alisis de la telemetr'ia \cite{azevedo_applying_2012}. Se consideraron dos casos reales de las anomal'ias de sat'elites en misiones espaciales de Brasil, lo que permiti'o evaluar y comparar la eficacia de dos algoritmos de agrupamiento, \emph{K-means} y expectativa de maximizaci'on (\emph{Expectation Maximization-EM}). Se logr'o demostrar su eficacia en varios canales de telemetr'ia que tend'ian a ofrecer valores at'ipicos y, en estos casos, podr'ian apoyar a los operadores de sat'elites, permitiendo la anticipaci'on de anomal'ias. Sin embargo, para los problemas silenciosos, donde solo hab'ia una peque'na variaci'on en un solo canal, los algoritmos no eran tan eficientes.\\

Las t'ecnicas actuales de los modelos de seguimiento de detecci'on de ciclones y de mediciones \emph{in situ}, no proporcionan una verdadera cobertura global, a diferencia de las observaciones satelitales remotas. Sin embargo, es poco pr'actico usar una sola 'orbita satelital para la detecci'on y seguimiento de estos eventos de una manera continua, debido a la cobertura espacial y temporal limitada. Una soluci'on para aliviar tales problemas persistentes es la de utilizar los datos de los sensores desde m'ultiples sat'elites en 'orbita. Este enfoque aborda los retos 'unicos asociados con el descubrimiento de conocimiento y miner'ia de flujos de datos por sat'elites heterog'eneos. Dicha orientaci'on consiste en tres componentes principales \cite{ho_automated_2008}: la extracci'on de caracter'isticas de cada medici'on del sensor, un clasificador para el descubrimiento del conjunto de ciclones, y el intercambio de conocimientos entre las distintas mediciones de los sensores remotos basados en un filtro lineal de \emph{Kalman} para el seguimiento de predicci'on de ciclones. Los resultados experimentales sobre datos hist'oricos de huracanes demuestran el rendimiento superior de este enfoque en comparaci'on con otros trabajos\cite{Dvorak_1984,Sinclair1997}.\\

Otro tipo de aplicaciones se han mostrado tambi'en en los sat'elites de transmisi'on de televisi'on \cite{li_application_2009} (TV \emph{broadcasting}). Con el fin de dar sentido a los datos alarmantes en el sistema de monitoreo del sat'elite, se introdujo un algoritmo mejorado para descubrir los episodios m'as frecuentes en la televisi'on de la red de monitoreo de radiodifusi'on por sat'elite. Los episodios frecuentes dentro de una determinada escala de los datos alarmantes se extraen con el fin de resumir el estado de los modelos de alarmas.\\

La miner'ia de datos espacial es la extracci'on de conocimiento impl'icito, las relaciones espaciales u otros patrones que no est'an almacenados de forma expl'icita en la base de datos espacial. Bajo este enfoque se coloca la derivaci'on de informaci'on de datos espaciales. Las coordenadas geogr'aficas de los ``\emph{Hot Spots}'' en las regiones de incendios forestales, que se extraen de las im'agenes de los sat'elites, son estudiadas y utilizadas en la detecci'on de los posibles puntos de fuego (\emph{hot spots}) \cite{tay_spatial_2003}. Dentro de las aplicaciones se encontraron que las falsas alarmas pueden ocurrir en los puntos de acceso derivados. Dada que esta informaci'on falsa puede ser identificada mediante la comparaci'on del resplandor detectado en varias bandas; se utiliz'o la agrupaci'on y la transformaci'on de \emph{Hough} para determinar los patrones regulares en los puntos de acceso derivados y clasificarlos como falsas alarmas. Esta implementaci'on demuestra la aplicaci'on de la miner'ia de datos espacial para reducir falsas alarmas del conjunto de puntos obtenidos a partir de las im'agenes.\\

Por 'ultimo, esta investigaci'on realiza un an'alisis de Big Data basado en los datos de los sat'elites de posicionamiento. Se presenta el desarrollo y el uso de un nuevo entorno de modelado distribuido de riesgo geol'ogico para el an'alisis y la interpretaci'on de los conjuntos de datos de terremotos de gran escala \cite{Guo2005}. Este trabajo se da en un entorno anal'itico distribuido de tiempo real donde los an'alisis y las simulaciones est'an estrechamente acopladas, integrando implementaciones de miner'ia de im'agenes de alto rendimiento, las cuales se ejecutan en servidores dedicados. Se logr'o simular terremotos en una escala micro y macro basados en im'agenes (\emph{imageodesy}) y en los datos hist'oricos de estos.

\clearpage