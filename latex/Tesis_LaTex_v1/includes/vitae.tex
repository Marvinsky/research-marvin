%Empieza configuracion
\setstretch{1.0}
\titleformat{\chapter}{\Huge\bfseries}{\thechapter}{0 pt}{\rule{340 pt}{3 pt}\\}
\titlespacing{\chapter}{100 pt}{-25 pt}{40 pt}[10 pt]	
\pagestyle{fancy}
\fancyhead[RO,RE]{\thepage}
\fancyfoot[CO,CE]{}
%Termina configuracion

\chapter*{Vitae}
\addcontentsline{toc}{chapter}{Vitae}
\setstretch{1.5} %Regresa el interlineado a 1.5

\normalsize
\noindent
Julio Cesar Roa Gil naci'o en Ibagu'e, Colombia el 25 de julio de 1987. Realiz'o sus estudios b'asicos en el Colegio Tolimense de Ibagu'e, Tolima e hizo sus estudios profesionales en la Universidad de Ibagu'e, donde obtuvo el t'itulo de Ingeniero de Sistemas en noviembre de 2010. Hizo parte de AIESEC, c'apituo Tolima, donde fue cordinador de Gesti'on del Conocimiento o \emph{Knowledge Management} (KM). Trabaj'o como Coordinador del proyecto ``Tolima Digital'' para el municipio de Mariquita, Tolima y desarrollador de los mapas de interactivos de la conectividad de la ciudad de Ibagu'e y del departamento del Tolima. Realiz'o un intercambio profesional en Belo Horizonte, Brasil, durante una a'no en la empresa Sydle, considerada una de las mejores empresas para trabajar en Tecnolog'ias de la informaci'on (TI) de acuerdo a \emph{Great Place To Work Institute}. All'i mismo desarrollo habilidades en bases de datos Oracle, siendo Administrador de Bases de Datos Junior o \emph{Database Administrator} (DBA) y adquiri'o las competencias del lenguaje Portugu'es. Trabaj'o como desarrollador para un proyecto de integraci'on, especificamente de ETL (\emph{Extract, Transform and Load}) para un importante banco de Colombia, en la ciudad de Bogot'a. Actu'o como lider t'ecnico, DBA, dise'nador y desarrollador para un proyecto de modernizaci'on de un sistema transaccional de cajeros autom'aticos para uno de los mayores grupos bancarios de Colombia. En el ITESM Campus Guadalajara, concluy'o sus estudios de Maestr'ia en Ciencias de la Computaci'on en diciembre de 2014.