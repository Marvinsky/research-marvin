%Empieza configuracion
\setstretch{1.0}
\titleformat{\chapter}{\Huge\bfseries}{\thechapter}{0 pt}{\rule{340 pt}{3 pt}\\}
\titlespacing{\chapter}{100 pt}{-25 pt}{40 pt}[10 pt]	
\pagestyle{fancy}
\fancyhead[RO,RE]{\thepage}
\fancyfoot[CO,CE]{}
%Termina configuracion

\chapter*{Resumen}
\addcontentsline{toc}{chapter}{Resumen}
\setstretch{1.5} %Regresa el interlineado a 1.5


\normalsize
\noindent El fen'omeno de Big Data ha impulsado durante los 'ultimos a'nos una revoluci'on en torno a los datos y su ventaja competitiva en el 'ambito empresarial y cient'ifico a trav'es de su an'alisis.
Cuando hablamos de Big Data impl'icitamente nos referimos a grandes vol'umenes de informaci'on, con una gran variedad de fuentes de informaci'on y que se generan a grandes velocidades.  Dentro de esta gran variedad de fuentes de informaci'on podemos encontrar las redes sociales, las bases de datos tradicionales, los Data Warehousing, los sensores de diferentes dispositivos, los sat'elites, entre otros.\\

En este trabajo, la principal fuente de informaci'on que se usar'a es la del conjunto de sat'elites rusos que componen el sistema Glonass, la cual se emplear'a para el an'alisis de Big Data, y su posterior visualizaci'on. El sistema Glonass, es uno de los tres principales sistemas de posicionamiento a nivel mundial; otros dos m'as conocidos son el sistema GPS de procedencia estadounidense y Galileo de procedencia europea.
La propuesta de tesis est'a basada en implementar un m'etodo de extracci'on en tiempo real y almacenamiento de datos, como tambi'en, en realizar un an'alisis e implementaci'on de Big Data a trav'es de diferentes procesos de miner'ia de datos, como el agrupamiento, a partir de los archivos de datos obtenidos de los distintos transmisores del sistema de posicionamiento ruso Glonass.\\

Los resultados obtenidos han sido la creaci'on de una aplicaci'on ETL para el procesamiento de los datos que funciona tanto en modo secuencial como en paralelo para optimizar el uso de los recursos de hardware, y as'i mismo el descubrimiento de patrones que se encontraban en el meta-modelo de la base de datos. Tambi'en el meta-modelo permiti'o responder a diferentes preguntas sobre los datos almacenados, como la cantidad de fallas y anomal'ias encontradas.

\clearpage