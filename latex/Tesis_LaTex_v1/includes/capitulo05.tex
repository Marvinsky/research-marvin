%Empieza configuracion de capitulo
\setstretch{1.0}
\titleformat{\chapter}[block]{\Large\bfseries}{CAP'ITULO \Huge\thechapter\vspace{25 pt}}{0 pt}{\\\fontsize{26}{36}\selectfont}
\titlespacing{\chapter}{0 pt}{30 pt}{50 pt}[0 pt]
\titleformat{\section}{\Large\bfseries}{\thesection}{0 pt}{\hspace{30 pt}}
\titleformat{\subsection}{\large\bfseries}{\thesubsection}{0 pt}{\hspace{30 pt}}
\pagestyle{fancy}
\fancyhead[LO,LE]{\footnotesize\emph{\leftmark}}
\fancyhead[RO,RE]{\thepage}
\fancyfoot[CO,CE]{}
%Termina configuracion de capitulo

\chapter{Experimentos} %Cambia al nombre de tu capitulo
\setstretch{1.5} %Regresa el interlineado a 1.5
\normalsize
\noindent
En el presente cap'itulo se describir'an los experimentos realizados con los \emph{Observation Files}, la base de datos, y el ambiente en el cual fueron desarrollados.\\

\section{Ambiente de pruebas}
\noindent
A continuaci'on se describen las principales caracter'isticas del equipo port'atil o \emph{Workstation} en los cuales se realizaron los experimentos.\\

\begin{table}[H]
\begin{center}
\scalebox{0.9}{	
\begin{tabular}{| l || l | }
\hline
\textbf{Item} & \textbf{Especificaci'on} \\ \hline \hline
Procesador & Procesador Intel Quad Core i7 4700HQ 4GHz\\ \hline
Sistema operativo & Windows 8\\ \hline
Chipset & Intel HM77 Chipset\\ \hline
Memoria Ram & 16 GB, DDR3 1800 MHz SDRAM\\ \hline
Panel TFT-LCD & C17.3" 16:9 FHD EWV LED Backlight\\ \hline
Tarjeta Gr'afica & NVIDIA GeForce GTX 770M 3GB GDDR5 VRAM \\ \hline
Almacenamiento & 1TB HDD 5400 RPM + 256GB SDD\\ \hline
\end{tabular}}
\end{center}
\caption{Especificaciones \emph{Workstation} Asus G750J }
\end{table}

\begin{table}[H]
\begin{center}
\scalebox{0.9}{	
\begin{tabular}{| l || l | }
\hline
\textbf{Item} & \textbf{Especificaci'on} \\ \hline \hline
CUDA Cores &  1536\\ \hline
Base Clock (MHz) & 1046\\ \hline
Boost Clock (MHz) & 1085\\ \hline
Texture Fill Rate (billion/sec) & 134\\ \hline
Velocidad de la memoria & 7.0 Gbps\\ \hline
Configuraci'on est'andar de la memoria & 2048 MB \\ \hline
Interface de memoria & GDDR5\\ \hline
Ancho de banda de memoria (GB/seg) & 224.3 \\ \hline
\end{tabular}}
\end{center}
\caption{Especificaciones tarjeta gr'afica, NVIDIA GeForce GTX 770M }
\end{table}

Se hizo uso de un disco duro externo para almacenar los archivos descargados. En la siguiente tabla se muestran sus especificaciones t'ecnicas:

\begin{table}[H]
\begin{center}
\scalebox{0.9}{	
\begin{tabular}{| l || l | }
\hline
\textbf{Item} & \textbf{Especificaci'on} \\ \hline \hline
Capacidad &  2TB\\ \hline
Interfaz & USB 2.0 y USB 3.0\\ \hline
Sistema operativo & Windows XP o superior\\ 
& Max OS X 10.4.6 Tiger o superior\\
& 10.5 Leopard o 10.6 Snow Leopard (32-bit kernel)\\ \hline
\end{tabular}}
\end{center}
\caption{Especificaciones disco duro externo, Seagate GoFlex Desk }
\end{table}

Referente a la parte de software, se utilizaron las siguientes herramientas:

\begin{table}[H]
\begin{center}
\scalebox{0.9}{	
\begin{tabular}{| l || l | l |}
\hline
\textbf{Software} & \textbf{Descripci'on} & \textbf{Versi'on} \\ \hline \hline
Eclipse IDE &  Entorno de desarrollo y ejecuci'on de pruebas (JAVA) & Luna\\ \hline
MySQL Workbench & Entorno de desarrollo y ejecuci'on de pruebas (SQL) & 6.1\\ \hline
VisualVM & \emph{Profiler} para la \emph{Java Virtual Machine} (JVM) & 1.3.8\\ \hline
MySQL Server & Servidor de Base de datos relacionales & 5.6.19\\ \hline
\end{tabular}}
\end{center}
\caption{Herramientas de software }
\end{table}

\section{Especificaciones generales}
\noindent
Las siguientes especificaciones aplican para todas las pruebas hechas con los datos de los archivos descargados.

\begin{table}[H]
\begin{center}
\scalebox{0.9}{	
\begin{tabular}{| l || l | }
\hline
\textbf{Item} & \textbf{Especificaci'on} \\ \hline \hline
Cantidad de archivos a procesar & 40775\\ \hline
Tama'no de los datos & 80.9 Gigabytes\\ \hline
Porcentaje m'aximo de error & 2\\ \hline
N'umero de ejecuciones & 3\\ \hline
\end{tabular}}
\end{center}
\caption{Especificaciones generales para las pruebas}
\end{table}

Las m'etricas que se utilizaron en el desarrollo de los experimentos se detallan a continuaci'on:

\begin{table}[H]
\begin{center}
\scalebox{0.9}{	
\begin{tabular}{| l || l | }
\hline
\textbf{M'etrica} & \textbf{Especificaci'on} \\ \hline \hline
Cantidad de archivos procesados con 'exito & n'umero natural\\ \hline
Cantidad de archivos procesados con fallas & n'umero natural\\ \hline
Tiempo de ejecuci'on & segundos, minutos y horas\\ \hline
\end{tabular}}
\end{center}
\caption{M'etricas generales para las pruebas}
\end{table}

\section{Pruebas}
\noindent
\begin{enumerate}
\item \textbf{Secuencial:} Consiste en ejecutar la aplicaci'on RINEX ETL de manera secuencial o con un 'unico hilo de ejecuci'on.
\item \textbf{Paralelo con procesadores:} Consiste en ejecutar la aplicaci'on RINEX ETL de manera paralela a trav'es de hilos, los cuales hacen uso de los \emph{cores} de los procesadores. 
Esta prueba se realiz'o con:
\begin{itemize}
\item 8 \emph{Threads}.
\item 10 \emph{Threads}.
\item 16 \emph{Threads}.
\item 32 \emph{Threads}.
\item 50 \emph{Threads}.
\end{itemize}
\item \textbf{Clustering:} Consiste en ejecutar la aplicaci'on \emph{Weka} y realizar la conexi'on con la base de datos del meta-modelo. Despu'es se realiz'o una consulta a la base de datos con la cual se cargaron los datos en la herramienta de miner'ia de datos y se ejecut'o el proceso de \emph{Clustering} con el objetivo de encontrar alguna anomal'ia en el \emph{Loss of Lock Indicator (LLI)} o alg'un patr'on sobre los datos descargados y de los cuales podamos inferir e intepretar posibles mejoras. Para cumplir este objetivo se utiliz'o el algoritmo de \emph{K-Means}, con el cual se definieron 4 \emph{clusters}.\\
\item \textbf{Consultas SQL:} A trav'es del lenguaje SQL se determinaron los siguientes objetivos:
\begin{itemize}
\item Identificar la cantidad de fallas en las 'epocas (\emph{epoch})
\item Identificar la cantidad de posibles deslizamientos de ciclo para el tipo de observaci'on L1
\item Identificar la cantidad de posibles deslizamientos de ciclo para el tipo de observaci'on L2
\item Determinar la frecuencia de los posibles deslizamientos de ciclo para el tipo de observaci'on L1
\item Determinar la frecuencia de los posibles deslizamientos de ciclo para el tipo de observaci'on L2
\end{itemize}

\end{enumerate}

\clearpage