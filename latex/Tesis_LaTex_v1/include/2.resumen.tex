%Empieza configuracion
\setstretch{1.0}
\titleformat{\chapter}{\Huge\bfseries}{\thechapter}{0 pt}{\rule{340 pt}{3 pt}\\}
\titlespacing{\chapter}{100 pt}{-25 pt}{40 pt}[10 pt]	
\pagestyle{fancy}
\fancyhead[RO,RE]{\thepage}
\fancyfoot[CO,CE]{}
%Termina configuracion

\chapter*{Resumen}
\addcontentsline{toc}{chapter}{Resumen}
\setstretch{1.5} %Regresa el interlineado a 1.5


\normalsize
\noindent
Esta tesis presenta un metodo formal para verificar planes de Internet basados en reglas, mediante Logica Temporal Lineal (LTL) y para determinar si hay conflicto entre las reglas dentro del plan. Los planes de Internet basados en reglas son formalizados y especificados utilizando nuestro modelo propuesto en Promela. \\

Escenarios actuales y futuros fueron tomados del dominio publico y utilizados para validar y explicar nuestro modelo propuesto. Todos los escenarios identificados fueron exitosamente modelados y verificados utilizando nuestro modelo en Promela y formulas LTLs introducidas como parte de nuestra tesis. \\

Los resultados obtenidos, demuestran que Model Checking es una herramienta v'alida y factible para modelar y verificar los planes de Internet basados en reglas.
\\


\clearpage