%Empieza configuracion de capitulo
\setstretch{1.0}
\titleformat{\chapter}[block]{\Large\bfseries}{CHAPTER \Huge\thechapter\vspace{25 pt}}{0 pt}{\\\fontsize{26}{36}\selectfont}
\titlespacing{\chapter}{0 pt}{30 pt}{50 pt}[0 pt]
\titleformat{\section}{\Large\bfseries}{\thesection}{0 pt}{\hspace{30 pt}}
\titleformat{\subsection}{\large\bfseries}{\thesubsection}{0 pt}{\hspace{30 pt}}
\pagestyle{fancy}
\fancyhead[LO,LE]{\footnotesize\emph{\leftmark}}
\fancyhead[RO,RE]{\thepage}
\fancyfoot[CO,CE]{}
%Termina configuracion de capitulo

\chapter{Conclusion} %Cambia al nombre de tu capitulo
\setstretch{1.5} %Regresa el interlineado a 1.5

\normalsize
\section{Summary of Contributions}
\noindent
In this thesis, we provided a framework to formally specify and verify rule-based Internet plans. The provided framework includes, a Promela Model to specify the plans and a set of four Linear Temporary Logic (LTL) properties to be used to verify the model. \\

The proposed Promela model can be used to unambiguously model rule-based Internet plans including a number of conditions such as Location, Application, Device and Time-of Day and different actions to be applied when the rule runs out of quota. \\

This thesis, identified two major LTL operators, Always ($\square$) and Eventually ($\lozenge$), which were used in all the use-cases; while the Until (U) LTL operator was applied only in one use-case.  \\

The real and future use-cases scenarios provided based on Internet Plans offered in the public domain, can be used as a future benchmark for other models. \\

Finally, it was proved that Model Checking is a feasible approach to detect conflicts between rules within Internet plans through the scenarios used. \\

\section{Future research}
\noindent
In Chapter 3, we introduced our proposed Promela model including several conditions and actions. Our model may be extended by adding other conditions to support use-cases not considered or offered yet. Similarly, the four LTL properties, may be extended accordingly. \\

We are able to conclude that Model Checking is a feasible and powerful tool to verify rule-based Internet plans and identify conflicts between their rules, per the experimental results. But since we have not tried our Model on a Plan with many rules, we have not challenged our model from a stressful verification perspective -- as more than 20 rules within only one plan would be unrealistic for now. Our future work includes more thorough experiments with numerous rules. \\

Other Model Checking tools could be considered to compare time or space requirements between them. \\

\clearpage
\noindent