%% abtex2-modelo-include-comandos.tex, v-1.9.5 laurocesar
%% Copyright 2012-2015 by abnTeX2 group at http://www.abntex.net.br/ 
%%
%% This work may be distributed and/or modified under the
%% conditions of the LaTeX Project Public License, either version 1.3
%% of this license or (at your option) any later version.
%% The latest version of this license is in
%%   http://www.latex-project.org/lppl.txt
%% and version 1.3 or later is part of all distributions of LaTeX
%% version 2005/12/01 or later.
%%
%% This work has the LPPL maintenance status `maintained'.
%% 
%% The Current Maintainer of this work is the abnTeX2 team, led
%% by Lauro César Araujo. Further information are available on 
%% http://www.abntex.net.br/
%%
%% This work consists of the files abntex2-modelo-include-comandos.tex
%% and abntex2-modelo-img-marca.pdf
%%

% ---
% Este capítulo, utilizado por diferentes exemplos do abnTeX2, ilustra o uso de
% comandos do abnTeX2 e de LaTeX.
% ---
 
\chapter{Greedy Heuristic Selection}\label{ghs}

\chapterprecis{The purpose of this section is to introduce the meta$-$reasoning proposed.}\index{sinopse de capítulo}

% ---
\section{Problem Formulation}
% ---
When solving $\triangledown$ using the consistent heuristic function $h_{max}(\zeta\sp{'})$ for $\zeta\sp{'} \subseteq \zeta$, A$\sp{*}$ expands in the worst case $J(\zeta\sp{'}, \triangledown)$ nodes, where

\begin{equation}
J(\zeta\sp{'},\triangledown) = |\{s \in V | f_{max}(s,\zeta\sp{'} \leq C\sp{*})\}|
\label{eq:eq_size_search_tree_1}
\end{equation}

\begin{equation}
J(\zeta\sp{'},\triangledown) = |\{s \in V | h_{max}(s,\zeta\sp{'} \leq C\sp{*}) - g(s)\}|
\label{eq:eq_size_search_tree_2}
\end{equation}

We present a greedy algorithm for approximately solving the following optimization problem,

\begin{equation}
\begin{split}
\textbf{minimize}_{\zeta\sp{'} \in 2\sp{|\zeta|}}J(\zeta\sp{'}, \triangledown) \\
\textbf{subject to} |\zeta\sp{'} = N|
\end{split}
\end{equation}

Where $N$ could be determined by a hard constraint such as the maximum number of \texttt{PDBs} one can store in memory.

% ---
\section{GHS Algorithm}
% ---
\noindent
Algorithm \ref{alg:ghs_algorithm} presents Greedy Heuristic Selection (\texttt{GHS}), an approximation algorithm for selecting a subset $\zeta\sp{'} \subseteq \zeta$.

The algorithm receives as input a planning problem $\triangledown$, a set of heuristics $\zeta$, a cardinality size $N$, and it returns a subset $\zeta\sp{'} \subseteq \zeta$ of size $N$. In each iteration \texttt{GHS} greedily selects from $\zeta$ the heuristic $h$ which will result in the largest reduction of the value of $J$ (line 3). \texttt{GHS} returns $\zeta\sp{'}$ once it has the desired cardinality size $N$.\\

\begin{algorithm}[H]
 
 \textbf{Input:} Problem $\triangledown$, set  of heuristics $\zeta$, cardinality $N$
 
 \textbf{Ouput:} heuristic subset $\zeta\sp{'} \subseteq \zeta$ of size $N$
 
 1:  $\zeta\sp{'} \leftarrow \emptyset$
 
 2:  \While{$|\zeta\sp{'}| < N$ do} {
 
 3:  $h \leftarrow \texttt{argmin}_{h \in \zeta} J(\zeta\sp{'} \cup \{h\}, \triangledown) $
 
 4: $\zeta\sp{'} \leftarrow \zeta\sp{'} \cup \{h\}$
 
 5: return $\zeta\sp{'}$
 }
 \caption{Greedy Heuristic Selection}
 \label{alg:ghs_algorithm}
\end{algorithm}

\subsection{GHS Approximation Analysis}
In the following analysis all heuristic functions are assumed to be consistent. We also assume that A$\sp{*}$ expands all nodes $n$ with $f(n) \leq C\sp{*}$ while solving $\triangledown$, as shown in Equation \eqref{eq:eq_size_search_tree_1}.

\section{Stratified Sampling (SS)}

\subsection{Type System}
\begin{figure}[htb]
\centering
\begin{forest}
 [\usebox\myboxc \hspace*{1.4in} \usebox\myboxb]
\end{forest}
\caption{The heuristic value is the position of the empty space in a Specific state.} \label{fig:type_system}
\end{figure}

% print types

\begin{forest}
 [\usebox\myboxcenter]
\end{forest}

\begin{forest}
 [\usebox\myboxcornerone]
\end{forest}

\begin{forest}
 [\usebox\myboxmediumleft \hspace*{0.2in} \usebox\myboxmediumup]
\end{forest}

\begin{forest}
 [\usebox\myboxcornerthree \hspace*{0.2in} \usebox\myboxcornertwo]
\end{forest}


\begin{forest}
 [\usebox\myboxmediumdown \hspace*{0.2in} \usebox\myboxmediumright]
\end{forest}


\begin{forest}
 [\usebox\myboxcornerfour]
\end{forest}

\clearpage