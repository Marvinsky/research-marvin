%% abtex2-modelo-include-comandos.tex, v-1.9.5 laurocesar
%% Copyright 2012-2015 by abnTeX2 group at http://www.abntex.net.br/ 
%%
%% This work may be distributed and/or modified under the
%% conditions of the LaTeX Project Public License, either version 1.3
%% of this license or (at your option) any later version.
%% The latest version of this license is in
%%   http://www.latex-project.org/lppl.txt
%% and version 1.3 or later is part of all distributions of LaTeX
%% version 2005/12/01 or later.
%%
%% This work has the LPPL maintenance status `maintained'.
%% 
%% The Current Maintainer of this work is the abnTeX2 team, led
%% by Lauro César Araujo. Further information are available on 
%% http://www.abntex.net.br/
%%
%% This work consists of the files abntex2-modelo-include-comandos.tex
%% and abntex2-modelo-img-marca.pdf
%%

% ---
% Este capítulo, utilizado por diferentes exemplos do abnTeX2, ilustra o uso de
% comandos do abnTeX2 e de LaTeX.
% ---

\externaldocument[I-]{chapter03}

\chapter{Empirical Evaluation}\label{Empirical Evaluation}

\chapterprecis{The purpose of this section is to present how our meta$-$reasoning is applied when solving \texttt{IPC} domain problens.}\index{sinopse de capítulo}

\texttt{RGHS} is guaranteed to find a near$-$optimal heuristic subset to guide the A$\sp{*}$ search granted that it is able to compute the objective function of interest. Thus, the practical effectiveness of \texttt{RGHS} depends on its ability of finding good approximations $\hat{J}$ and $\hat{T}$. Moreover, \texttt{RGHS}'s effectiveness depends on the value of $N$ and on the quality of the set of heuristics $\zeta$. In order to verify its practical effectivess, we have implemented \texttt{RGHS} in Fast Downward \cite{helmert2006fast} and tested the A$\sp{*}$ performance using subsets of heuristics selected by \texttt{RGHS} while minimizing different objective funcions.\\

We run two sets of experiments. In the first set we verify whether the approximations $\hat{J}$ and $\hat{T}$ provided by \texttt{CS} and \texttt{SS} allow \texttt{RGHS} to make near$-$optimal subset selections. In the second set of experiments we test the effectiveness of \texttt{RGHS} by measuring the total number of problem instances solved by A$\sp{*}$ using a heuristic subset selected by \texttt{RGHS}.\\

In our experiments we implemented an approximation procedure to automatically choose the subset size selected by \texttt{RGHS}, which we now explain. Initially we fix $N$ to a value (in our experiments we use $N = 25$). Then, in every iteration, when computing $M_{i}$, \texttt{RGHS} removes from $\zeta$ the heuristics that cannot help reducing $\varphi$. That is, in iteration $i$, all heuristics in $\zeta \backslash \zeta_{i-1}\sp{'}$ that cannot reduce $\varphi$ when combined with $\zeta_{i-1}\sp{'}$ are removed from $\zeta$. \texttt{RGHS} stops if $\zeta$ is empty and returns the current subset $\zeta\sp{'}$ with $|\zeta\sp{'}|<N$. Likewise, should heuristic $h$ improves the objective function being optimized, then we also allow \texttt{RGHS} to add $h$ to $\zeta\sp{'}$ even if $|\zeta\sp{'}| > N$. We observed in preliminary experiments that \texttt{RGHS} is robust to the initial value of $N$, as the total number of problems solved by A$\sp{*}$ while guided by a heuristic subset selected by \texttt{RGHS} varied little with $N$.\\

In all our experiments we use a type$-$system that assigned the same type for a node with the same $f-$value. Such a type system has shown to be effective in guiding \texttt{SS} to produce accurate tree size predictions in other application domains \cite{lelis2013predicting,lelis2014memory}.\\

We ran our experiments on the 2011 International Planning Competition (\texttt{IPC}) instances. We used the 2011 instances instead of the 2014 instances because the former do not have problems with conditional effects, which are currently not handled by \textbf{PDB} heuristics. All experiments are run on $2.67$ GHz machines with 4GB, and are limited to 1,800 seconds of running time.\\

% ---
\section{Empirical Evaluation of $\hat{J}$ and $\hat{T}$}
\noindent
% ---
We test whether the approximation $\hat{J}$ provided by \texttt{CS} and \texttt{SS} allows \texttt{RGHS} to make near$-$optimal subset selections. This test is made by comparing $J(\zeta\sp{'})$ with $J(\zeta)$, which is minimal. The condition $J(\zeta\sp{'}) \leq \alpha \cdot J(\zeta)$, is sufficient to show that $J(\zeta\sp{'})$ is within $\alpha$ times optimal with respect to all subsets of any size, for some constant $\alpha$. We are particularly interested in observing if $J(\zeta\sp{'})$ is within 1.36 of $J(\zeta)$.\\

In contrast with objective function $J$, there is no easy way to find the minimum of $T$ for a subset of fixed size in general. We experiment then with the special case in which all heuristics in $\zeta$ have the same evaluation time. This way we are able to test whether the estimates $\hat{T}$ are allowing \texttt{RGHS} to make near$-$optimal subset selections while minimizing the A$\sp{*}$ running time. This is because by only selecting heuristics which have the same evaluation time, if \texttt{RGHS} is making near$-$optimal subset selections with respect to $J$, then \texttt{RGHS} must also be making near$-$optimal subset selectioons with respect to $T$ for a fixed subset size (See Theorem \ref{th:theorem_two}).\\

We collect values of $J(\zeta)$ and $J(\zeta\sp{'})$ as follows. For each problem instance $\nabla$ in our test set we generate a set of \texttt{PDB} heuristics using the \texttt{GA-PDB} algorithm \cite{edelkamp2007automated} as described by \cite{BarleySantiagoOver} $-$ we call each \texttt{PDB} generated by this method a \texttt{GA-PDB}. We chose to use \texttt{GA-PDBs} in this experiment because they all have nearly the same evaluation time and will allow us to verify whether \texttt{RGHS} is making near$-$optimal selections not only when minimizing $J$ but also when minimizing $T$, as explained above. The number of \texttt{GA-PDBs} generated is limited in this experiment by 1,200 seconds and 1GB of memory. Also, all \texttt{GA-PDBs} we generate have 2 millions entries each. The \texttt{GA-PDBs} generated form out $\zeta$ set. \texttt{RGHS} then selects a subset $\zeta\sp{'}$ of $\zeta$. Finally, we use $h_{max}(\zeta\sp{'})$ and $h_{max}(\zeta)$ to independently try to solve $\nabla$. We call the system which uses A$\sp{*}$ with $h_{max}(\zeta)$ the \texttt{Max} approach. For \texttt{RGHS} we allow 600 seconds for selecting $\zeta\sp{'}$ and for running A$\sp{*}$ with $h_{max}(\zeta\sp{'})$, and for \texttt{Max} we allow 600 seconds for running A$\sp{*}$ with $h_{max}(\zeta)$. Since we used 1,200 seconds to generate the heuristics, both \texttt{Max} and \texttt{RGHS} were allowed 1,800 seconds in total for solving each problem. In this experiment we test both \texttt{CS} and \texttt{SS}.\\

In this experiment we refer to the approach that runs A$\sp{*}$ guided by a heuristic subset selected by \texttt{RGHS} using \texttt{CS} as \texttt{RGHS+CS}. Similarly, we write \texttt{RGHS+SS} when \texttt{SS} is used as predictor to make the heuristic subset selection.\\

