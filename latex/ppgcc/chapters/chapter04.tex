%% abtex2-modelo-include-comandos.tex, v-1.9.5 laurocesar
%% Copyright 2012-2015 by abnTeX2 group at http://www.abntex.net.br/ 
%%
%% This work may be distributed and/or modified under the
%% conditions of the LaTeX Project Public License, either version 1.3
%% of this license or (at your option) any later version.
%% The latest version of this license is in
%%   http://www.latex-project.org/lppl.txt
%% and version 1.3 or later is part of all distributions of LaTeX
%% version 2005/12/01 or later.
%%
%% This work has the LPPL maintenance status `maintained'.
%% 
%% The Current Maintainer of this work is the abnTeX2 team, led
%% by Lauro César Araujo. Further information are available on 
%% http://www.abntex.net.br/
%%
%% This work consists of the files abntex2-modelo-include-comandos.tex
%% and abntex2-modelo-img-marca.pdf
%%

% ---
% Este capítulo, utilizado por diferentes exemplos do abnTeX2, ilustra o uso de
% comandos do abnTeX2 e de LaTeX.
% ---

\externaldocument[I-]{chapter03}

\chapter{Empirical Evaluation}\label{ch:empirical_evaluation}

\chapterprecis{The purpose of this section is to present how our meta$-$reasoning is applied when solving \texttt{IPC} domain problens.}\index{sinopse de capítulo}

\texttt{RGHS} is guaranteed to find a near$-$optimal heuristic subset to guide the A$\sp{*}$ search granted that it is able to compute the objective function of interest. Thus, the practical effectiveness of \texttt{RGHS} depends on its ability of finding good approximations $\hat{J}$ and $\hat{T}$. Moreover, \texttt{RGHS}'s effectiveness depends on the value of $N$ and on the quality of the set of heuristics $\zeta$. In order to verify its practical effectivess, we have implemented \texttt{RGHS} in Fast Downward \cite{helmert2006fast} and tested the A$\sp{*}$ performance using subsets of heuristics selected by \texttt{RGHS} while minimizing different objective funcions.\\

We run two sets of experiments. In the first set we verify whether the approximations $\hat{J}$ and $\hat{T}$ provided by \texttt{CS} and \texttt{SS} allow \texttt{RGHS} to make near$-$optimal subset selections. In the second set of experiments we test the effectiveness of \texttt{RGHS} by measuring the total number of problem instances solved by A$\sp{*}$ using a heuristic subset selected by \texttt{RGHS}.\\

In our experiments we implemented an approximation procedure to automatically choose the subset size selected by \texttt{RGHS}, which we now explain. Initially we fix $N$ to a value (in our experiments we use $N = 25$). Then, in every iteration, when computing $M_{i}$, \texttt{RGHS} removes from $\zeta$ the heuristics that cannot help reducing $\varphi$. That is, in iteration $i$, all heuristics in $\zeta \backslash \zeta_{i-1}\sp{'}$ that cannot reduce $\varphi$ when combined with $\zeta_{i-1}\sp{'}$ are removed from $\zeta$. \texttt{RGHS} stops if $\zeta$ is empty and returns the current subset $\zeta\sp{'}$ with $|\zeta\sp{'}|<N$. Likewise, should heuristic $h$ improves the objective function being optimized, then we also allow \texttt{RGHS} to add $h$ to $\zeta\sp{'}$ even if $|\zeta\sp{'}| > N$. We observed in preliminary experiments that \texttt{RGHS} is robust to the initial value of $N$, as the total number of problems solved by A$\sp{*}$ while guided by a heuristic subset selected by \texttt{RGHS} varied little with $N$.\\

% Please add the following required packages to your document preamble:
% \usepackage{multirow}
\begin{table}[]
\centering
\caption{Ratios of the number of nodes expanded using $h_{max}(\zeta\sp{'})$ to the number of nodes expanded using $h_{max}(\zeta)$}
\label{my-label}
\begin{tabular}{lcccccc}
\hline
\multirow{2}{*}{Domain} & \multicolumn{2}{c}{\texttt{SS}} & \multicolumn{2}{c}{\texttt{CS}}   & \multirow{2}{*}{|$\zeta$|} & \multirow{2}{*}{n} \\ \cline{2-5}
                        & Ratio    & |$\zeta\sp{'}$|   & Ratio  & |$\zeta\sp{'}$| &                            &                    \\ \hline
Barman                  & 2.15     & 27.26       & 2.34   & 30.68           & 4790.16                    & 20                 \\
Elevators               & 1.09     & 18.00       & 1.04   & 20.00           & 155.00                     & 1                  \\
Floortile               & 1.01     & 45.86       & 1.01   & 41.14           & 136.57                     & 13                 \\
Openstacks              & 1.00     & 1.00        & 1.00   & 1.00            & 316.62                     & 12                 \\
Parking                 & 1.01     & 7.05        & 1.01   & 7.37            & 17.53                      & 18                 \\
Pegsol                  & 1.00     & 32.00       & 1.00   & 48.33           & 78.00                      & 2                  \\
Scanalyzer              & 1.01     & 37.29       & 1.07   & 21.43           & 52.14                      & 6                  \\
Tidybot                 & 1.03     & 4.06        & 1.00   & 8.19            & 3259.00                    & 15                 \\
Transport               & 7.33     & 9.70        & 1.45   & 12.90           & 156.50                     & 9                  \\
Visitall                & 1.00     & 95.00       & 1.02   & 48.00           & 218.00                     & 2                  \\
Woodworking             & 2.39     & 83.00       & 611.72 & 32.00           & 346.00                     & 5                  \\ \hline
\end{tabular}
\label{tb_one}
\end{table}

In all our experiments we use a type$-$system that assigned the same type for a node with the same $f-$value. Such a type system has shown to be effective in guiding \texttt{SS} to produce accurate tree size predictions in other application domains \cite{lelis2013predicting,lelis2014memory}.\\

We ran our experiments on the 2011 International Planning Competition (\texttt{IPC}) instances. We used the 2011 instances instead of the 2014 instances because the former do not have problems with conditional effects, which are currently not handled by \textbf{PDB} heuristics. All experiments are run on $2.67$ GHz machines with 4GB, and are limited to 1,800 seconds of running time.\\

% ---
\section{Empirical Evaluation of $\hat{J}$ and $\hat{T}$}
\noindent
% ---
We test whether the approximation $\hat{J}$ provided by \texttt{CS} and \texttt{SS} allows \texttt{RGHS} to make near$-$optimal subset selections. This test is made by comparing $J(\zeta\sp{'})$ with $J(\zeta)$, which is minimal. The condition $J(\zeta\sp{'}) \leq \alpha \cdot J(\zeta)$, is sufficient to show that $J(\zeta\sp{'})$ is within $\alpha$ times optimal with respect to all subsets of any size, for some constant $\alpha$. We are particularly interested in observing if $J(\zeta\sp{'})$ is within 1.36 of $J(\zeta)$.\\

In contrast with objective function $J$, there is no easy way to find the minimum of $T$ for a subset of fixed size in general. We experiment then with the special case in which all heuristics in $\zeta$ have the same evaluation time. This way we are able to test whether the estimates $\hat{T}$ are allowing \texttt{RGHS} to make near$-$optimal subset selections while minimizing the A$\sp{*}$ running time. This is because by only selecting heuristics which have the same evaluation time, if \texttt{RGHS} is making near$-$optimal subset selections with respect to $J$, then \texttt{RGHS} must also be making near$-$optimal subset selectioons with respect to $T$ for a fixed subset size (See Theorem \ref{th:theorem_two}).\\

We collect values of $J(\zeta)$ and $J(\zeta\sp{'})$ as follows. For each problem instance $\nabla$ in our test set we generate a set of \texttt{PDB} heuristics using the \texttt{GA-PDB} algorithm \cite{edelkamp2007automated} as described by \cite{BarleySantiagoOver} $-$ we call each \texttt{PDB} generated by this method a \texttt{GA-PDB}. We chose to use \texttt{GA-PDBs} in this experiment because they all have nearly the same evaluation time and will allow us to verify whether \texttt{RGHS} is making near$-$optimal selections not only when minimizing $J$ but also when minimizing $T$, as explained above. The number of \texttt{GA-PDBs} generated is limited in this experiment by 1,200 seconds and 1GB of memory. Also, all \texttt{GA-PDBs} we generate have 2 millions entries each. The \texttt{GA-PDBs} generated form out $\zeta$ set. \texttt{RGHS} then selects a subset $\zeta\sp{'}$ of $\zeta$. Finally, we use $h_{max}(\zeta\sp{'})$ and $h_{max}(\zeta)$ to independently try to solve $\nabla$. We call the system which uses A$\sp{*}$ with $h_{max}(\zeta)$ the \texttt{Max} approach. For \texttt{RGHS} we allow 600 seconds for selecting $\zeta\sp{'}$ and for running A$\sp{*}$ with $h_{max}(\zeta\sp{'})$, and for \texttt{Max} we allow 600 seconds for running A$\sp{*}$ with $h_{max}(\zeta)$. Since we used 1,200 seconds to generate the heuristics, both \texttt{Max} and \texttt{RGHS} were allowed 1,800 seconds in total for solving each problem. In this experiment we test both \texttt{CS} and \texttt{SS}.\\

In this experiment we refer to the approach that runs A$\sp{*}$ guided by a heuristic subset selected by \texttt{RGHS} using \texttt{CS} as \texttt{RGHS+CS}. Similarly, we write \texttt{RGHS+SS} when \texttt{SS} is used as predictor to make the heuristic subset selection.\\

Table \ref{tb_one} shows the average ratios of $J(\zeta\sp{'})$ to $J(\zeta)$ for both \texttt{SS} and \texttt{CS} in different problem domains. The value of $J$, for a given problem instance, is computed as the number of nodes expanded up to the largest $f-$layer which is fully expanded by all approaches tested (\texttt{Max, RGHS} using \texttt{SS} and \texttt{RGHS} using \texttt{CS}). We only present results for instances that are not solved during \texttt{RGHS}'s \texttt{CS} sampling process. The column $"n"$ shows the number of instances used to compute the averages of each row. We also show the average number of \texttt{GA-PDBs} generated (|$\zeta$|) and the average number of \texttt{GA-PDBs} selected by \texttt{RGHS} (|$\zeta\sp{'}$|). This experiment shows that for most of the problems \texttt{RGHS}, using either \texttt{CS} or \texttt{SS}, is selecting a near$-$optimal subset of $\zeta$. For example, in Tidybot \texttt{RGHS} selects only a few \texttt{GA-PDBs} out of thousands when using either \texttt{SS} or \texttt{CS}. Moreover, the resulting A$\sp{*}$ search tree size is on average at most 3$\%$ larger than optimal for \texttt{RGHS+SS}, and is optimal for \texttt{RGHS+CS}.\\

The exceptions in Table \ref{tb_one} are the ratios for Barman, Transport and Woodworking. In Transport \texttt{SS} has an average ratio of $7.33$ and \texttt{CS} of $1.45$. By looking at the ratios of \texttt{SS} for individual instances of \texttt{SS} for individual instances of Transport (results now show in Table \ref{tb_one}), we noticed that \texttt{SS} is able to make optimal selections for all but one of the 9 instances considered in this experiment. Since we do not know a priori what is the instance's optimal solution cost, \texttt{SS} samples nodes with $f-$values much larger than the instance's optimal solution cost. We believe that, in this particular instance of Transport, by sampling a portion of the state space that is not expanded during the actual A$\sp{*}$, \texttt{SS} is biasing the subset selection to select heuristics that do not contribute to reducing the actual A$\sp{*}$ search tree size.\\

The \texttt{SS}'s ability of sampling deep into the search space is not always harmful. For example, \texttt{SS} allows \texttt{RGHS} to make good selections for instances of the Woodworking domain. By contrast, \texttt{CS}'s systematic approach to sampling only allow a shallow sample of the A$\sp{*}$ search tree. As a result, \texttt{RGHS} makes a limited selection of heuristics to guide A$\sp{*}$ search. While \texttt{RGHS} using \texttt{SS} selects an average of 83 heuristics in Woodworking instances, \texttt{RGHS} using \texttt{CS} selects only an average of 32 heuristics. This difference on sampling strategies reflects on the number of problems solved by A$\sp{*}$. While \texttt{RGHS+SS} solves 15 instances of the Woodworking domain, \texttt{RGHS+CS} solves only 11. In total, out of the 280 instances of the \texttt{IPC} 2011 benchmark set, \texttt{RGHS+SS} solves 200 problem instances in this experiment, while \texttt{RGHS+CS} only solves 193 problem instances. (The numbers of instances solved are not shown in Table \ref{tb_one}).

\section{Comparison with Other Planning Systems}
\noindent
The objective of this second set of experiments is to teset the quality of the subset of heuristics \texttt{RGHS} selects while optimizing different objective functions. Our evaluation metric is coverage, \textsc{i.e.,} number of problems solved within a 1,800 second time limit. We note that the 1,800$-$second limit includes the time to generate $\zeta$, select $\zeta\sp{'}$, and run A$\sp{*}$ using $h_{max}(\zeta\sp{'})$. The $\zeta$ set of heuristics is composed of a number of different \texttt{GA-PDBs}, a \texttt{PDB} heuristic produced by the \texttt{iPDB} method \cite{haslum2007domain} and the \texttt{LM-Cut} heuristic. The generation of \texttt{GA-PDBs} is limited by 600 seconds and 1GB of memory. We use one fourth of 600 seconds to genetate \texttt{GA-PDBs} with each of the following number of entries: $\{2 \cdot 10\sp{3}, 2 \cdot 10\sp{4}, 2 \cdot 10\sp{5}, 2 \cdot 10\sp{6}\}$. Our approach allows one to generate up to thousands of \texttt{GA-PDBs}. For every problem instance, we use exactly the same $\zeta$ set for \texttt{Max} and all \texttt{RGHS} approaches.\\

\section{Systems Tested}
\noindent
\texttt{RGHS} is tested while minimizing the A$\sp{*}$ search tree size (\texttt{Size}) and the A$\sp{*}$ running time (\texttt{Time}). We also use \texttt{RGHS} to maximize the sum of heuristic values in the state space (\texttt{Sum}), as suggested by \cite{raynersss13}. \cite{raynersss13} assumed that one could uniformly sample states in the state space in order to estimte the sum of the heuristic values for a given heuristic subset. Since we are not aware of any method to uniformly sample the state space of domain$-$independent problems, we adapted the \cite{raynersss13}'s method by using \texttt{SS} to estimate the sum of heuristic values in the search tree rooted at $\nabla$'s start state. We write \texttt{Size + SS} to refer to the approach that used A$\sp{*}$ guided by a heuristic selected by \texttt{RGHS} while minimizing an estimate of the search tree size provided by \texttt{SS}. We follow the same pattern to name the other possible combinations of objective functions and prediction algorithms (\textsc{e.g.,}\texttt{Time + CS}).\\

In addition to experimenting with all combination of prediction algorithms (\texttt{CS} and \texttt{SS}) and objective functions (\texttt{Time}, \texttt{Size}), we also experiment with an approach that minimizes both the search tree size and the running time as follows. First we create a pool of heuristics $\zeta$ composed solely of \texttt{GA-PDB} heuristics, then we apply \texttt{RGHS} while minimizing tree size and using \texttt{SS} as predictor. As explained above, in this setting \texttt{RGHS} minimizes $J$ and $T$ simultaneaously, as all heuristics in $\zeta$ have the same evaluation time. We call the selection of a subset of \texttt{GA-PDBs} as the \textit{first selection}. Once the first selection is made, we test all possible combinations of the resulting $h_{max}(\zeta\sp{'})$ added to the \texttt{iPDB} and \texttt{LM-Cut} heuristics while minimizing the running time as estimated by \texttt{CS}$-$we call this step the \textit{second selection}. We call the overall approach \texttt{Hybrid}.\\

The intuition behind \texttt{Hibrid} is that we apply \texttt{RGHS} with its strongest settings, \textsc{i.e.,} according to Theorem \ref{th:theorem_two} \texttt{RGHS} makes selections that are within the 1.36 factor of optimal with respect to both $J$ and $T$ when selecting from a pool of heuristics with the same evaluation time. After such a selection is made, we reduce the size of the pool of heuristics from possible thousands to only tree (the maximum of a subset of the initial \texttt{GA-PDBs, iPDB}, and \texttt{LM-Cut}). With only three heuristics we are able to choose the exact combination that minimizes the A$\sp{*}$ running time the most. The reason we chose to use \texttt{SS} instead of \texttt{CS} for the first selection in \texttt{Hybrid} is that the former is able to make better subset selections in this setting, as suggested by the results discussed in the previous Part \ref{ch:rghs}. Finally, as we show below, \texttt{CS} is more effective if one is interested in minimizing the A$\sp{*}$ running time while selecting from a pool of heuristic with different evaluation times. That is why we use \texttt{CS} as predictor for the second selection in \texttt{Hybrid}.\\

We compare the coverage of the \texttt{RGHS} approaches with several other state$-$of$-$the$-$art planners. Namely, we experiment with RIDA$\sp{*}$ \cite{BarleySantiagoOver}, two variants of StoneSoup (StSp1 and StSp2) as described by \cite{nissim2011computing}, two versions of Symba (SY1 and SY2), and A$\sp{*}$ being independently guided by the maximum of all heuristics in $\zeta$ (\texttt{Max}), \texttt{iPDB, LM-cut} and \texttt{Merge $\&$ Shrink}(\texttt{M$\&$S}) \cite{nissim2011computing}.\\

The results are presented in Table