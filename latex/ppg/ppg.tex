
\documentclass[11pt,a4paper,oneside]{report}

\synctex=1

\usepackage{pslatex,palatino,avant,graphicx,color}
\usepackage[margin=2cm]{geometry}

\usepackage{mathptmx}       % selects Times Roman as basic font

\usepackage{helvet}         % selects Helvetica as sans-serif font
\usepackage{courier}        % selects Courier as typewriter font
\usepackage{type1cm}        % activate if the above 3 fonts are
                            % not available on your system

\usepackage{makeidx}         % allows index generation
\usepackage{graphicx}        % standard LaTeX graphics tool
                             % when including figure files
\usepackage{multicol}        % used for the two-column index
\usepackage{multirow}
\usepackage{booktabs}

\usepackage[bottom]{footmisc}% places footnotes at page bottom

\usepackage[utf8]{inputenc}
\inputencoding{utf8}


%\usepackage[round]{natbib}
\usepackage[square,sort]{natbib}
\bibliographystyle{unsrtnat}


\usepackage[utf8]{inputenc}
\usepackage[english]{babel}

\usepackage{longtable}
\usepackage{pdflscape}

%for formula references
\usepackage{amsmath}
\newtheorem{mydef}{Definition}

\usepackage{changes}
%triangledown
\usepackage{latexsym}
\usepackage{amssymb}
\usepackage{amsfonts}


%langle
\usepackage{scalerel}
\usepackage{graphicx}

%cite

%\usepackage[USenglish]{babel}
%\usepackage[nodayofweek,level]{datetime}
%\newcommand{\mydate}{\formatdate{4}{11}{2014}}

\makeindex

\begin{document}
\title{On Selecting Heuristics Functions for Domain-Independent Planning.}
\providecommand{\keywords}[1]{\textbf{keywords} #1}
\author{Student: Marvin Abisrror Zarate\\ Advisor: Levi Lelis \\
Departamento de Informática \\Universidade Federal de Vicosa \\Viçosa, Brazil}

\date{\color{black}July 2015}
\maketitle

\section{Short Paper}
\label{sec:1}
Selecting a subset of heuristics functions in order to solve Domain-Indepentent Planning problems using A* search algorithm is an approach that selects the most promising heuristics from a bunch of heuristics that were generated based on the concepts of reducing the search tree and combination of primitive heuristics \citep{BarleySantiagoOver}.\\

The approach we choose to select the most promising heuristics is Deterministic and is based on the ranking of heuristics according the number of nodes generated running the Stratification Sampling algorithm (SS). The SS system \citep{lelis2013predicting} is a stocastic approach that produces accurate results predicting the number of nodes expanded by IDA*. Nevertheless, we are aware that SS can not be easily adapted to A* because A*'s duplicate pruning makes it very difficult to predict how many nodes will occur at depth \textit{\textbf{d}} of A* search tree (the tree of nodes expanded by A*). That is the principal reason why we decided to create the concept of \textit{regret} which is the cost from not to choose the heuristic that generates the lowest number of nodes generated or the best heuristic. If the \textit{regret} for one heuristic is very close to zero when running A* then it is a promising heuristic.\\

This project is concerned with cost-optimal state-space planning using the A* algorithm \citep{hart1968formal}. The main aim of this research is to develop an strategy to make the A* search algorithm resolves problems quickly. The approach we are proposing can be applied for solving optimal domain problems in the Fast-Downward Planner \citep{helmert2006fast}. A few of these domains are: Elevators opt08 and opt11, Floortile opt11, Nomystery opt11, etc.
\\

The contribution we expect from our approach is that the problem of finding a subset of heuristics is determinated by the ranking of heuristics based on the number of nodes generated by SS influence in the ranking of heuristics based on the number of nodes generated by A*.
\newpage

%Imports the bibliography file "references.bib"
\bibliography{references}
%\bibliographystyle{references}

\end{document}

