
\documentclass[11pt,a4paper,oneside]{report}

\synctex=1

\usepackage{pslatex,palatino,avant,graphicx,color}
\usepackage[margin=2cm]{geometry}

\usepackage{mathptmx}       % selects Times Roman as basic font

\usepackage{helvet}         % selects Helvetica as sans-serif font
\usepackage{courier}        % selects Courier as typewriter font
\usepackage{type1cm}        % activate if the above 3 fonts are
                            % not available on your system

\usepackage{makeidx}         % allows index generation
\usepackage{graphicx}        % standard LaTeX graphics tool
                             % when including figure files
\usepackage{multicol}        % used for the two-column index
\usepackage{multirow}
\usepackage{booktabs}

\usepackage[bottom]{footmisc}% places footnotes at page bottom

\usepackage[utf8]{inputenc}
\inputencoding{utf8}

\usepackage[round]{natbib}
\bibliographystyle{unsrtnat}


\usepackage[utf8]{inputenc}
\usepackage[english]{babel}

\usepackage{longtable}
\usepackage{pdflscape}

%for formula references
\usepackage{amsmath}
\newtheorem{mydef}{Definition}


%\usepackage[USenglish]{babel}
%\usepackage[nodayofweek,level]{datetime}
%\newcommand{\mydate}{\formatdate{4}{11}{2014}}

\makeindex

\begin{document}
\title{Predicting the Number of Nodes Expanded by A*}
\providecommand{\keywords}[1]{\textbf{keywords} #1}
\author{Marvin Abisrror Zarate\\ Departamento de Informática \\Universidade Federal de Vicosa \\Viçosa, Brazil}

\date{\color{black}July 2015}
\maketitle

\abstract{Predict the time an algorithm will last to find a solution or which heuristic is better to use for an specific domain problem are questions we do when we start to planning the route to get the solution of the problem. There are instances that try to find their goal state could take too much time and people do not have enough time to spend watching if the algorithm found the solution. In this project we propose a new approach to make prediction about how long an algorithm last to find a solution, our hypotheses is based in the number of nodes the algorithm expands during the search. So, we make predictions about the number of nodes the algorithm will expand during the search until the solution is found in order to know how long the algorithm will last to solve the problem. We use the optimal problem domains from fast-downward planning system to applied our approach and the results are competitive respect other approachs.}

\keywords{Prediction; Artificial Intelligence; A*}
\section{Introduction}
\label{sec:1}
We chose A* as our algorithm that we will predict the number of nodes it expands for a specific instance. Our approach has two main stages which are: In the first stage we use the tree generated by Dijkstra algorithm to collect the number of nodes expanded by level. In the second stage we use Stratified Sampling (SS) to generate the distribution of the \textit{f-value} by level.  With these two stages completed for the problem we make predictions based on a mathematical formulas proposed here.\newline


Informed algorithms such as A* \citep{hart1968formal} is very important in AI because guide us in the resolution of interesting applications. This important algorithm has two elements: The heuristic value from a state \textbf{\textit{s}}, which give us an estimate of the remaining goal distance named as \textbf{\textit{h}}. Intuitively, nodes with lower heuristic value are more promising than nodes with higher heuristic value because they seem to be closer to the solution.  And the cost to go from the origin state to the current state named as \textbf{\textit{g}}. The sum of these elements \textbf{\textit{f = h + g}} is called cost function, which help us to prioritize node expansion during the search chosing the nodes that generate a close approximation to the solution path.
\newline

Dijkstra search Algorithm generate a \textbf{SPT}(\textit{shortest path tree}) with given source \textbf{\textit{s}} as root. Dijkstra is a greedy algorithm where we maintain two sets, one set contains nodes included in shortest path tree, and the other set includes nodes not yet included in shortest path tree. In each iteration of the algorithm, we find a node which is in the other set (not yet included) and has minimum distance from the start state which is added to the shortest path tree set. The heuristic value when Dijkstra is used is \textbf{\textit{0}} for all the nodes and as consequence the guide to get the solution is the cost to go from the parent node to their child node. Then, Dijkstra is not considered as informed algorithm, for our purpose we will use the information of the nodes expanded at each level by Dijkstra as our first term of our prediction formula.
\newline

Stratification Sampling (\textbf{SS})  is applied to the problem to predict the number of nodes expanded in the tree. \textbf{SS} works iteratively expanding without counting nodes that represent the same state because it chose randomly one of the repeated node that have more chance to be used and removing or avoiding the insertion of the other. This behavior does that \textbf{SS} overestimates the real number of nodes of the tree. We collect the nodes with \textit{f-value} less than or equal to the threshold \textbf{\textit{d}} which is equal to two times the heuristic value of the initial state: 
\[ 
d =2 \times (s*)
\]
In each level of the tree geneerated by \textbf{SS} we chose the nodes that meets the condition in order to generate our distribution.  We build our distribution based on \textbf{SS} because \citep{lelis2013predicting} have shown that \textbf{SS} produce much more accurate predictions of the search tree than competing schemes such as \textbf{CDP} \citep{zahavi2010predicting}.
\newline

The formula to predict the number of nodes expanded by A* are  related with the information from Dijkstra Algorithm and the distribution of \textit{f-value} from \textbf{SS}.

\section{Background}
We use as experiments the competitions domains problems from the fast-downward benchmarks. The planner contains satisficing and optimization track problems, the satisficing problem are so hard to work and also is difficicult to found an optimal solutions, so they could be used to research because we pretend to predict the number of nodes expanded in the tree but we won't found an optimal solutions and also it would skip our analysis using consistent heuristics. In this research we will use the problems that require optimization in order to test our approach. 

\subsection{Problem Domains}
Two of the optimal domain problems we have worked are barman and parking. The block domain contains easy instances that are very good to test our approach quickly.

\subsubsection{Barman}
This domain was created for the \textit{International Planning Competition (IPC)} in \textit{2011}. In this domain there is a robot barman that manipulates drink dispensers, glasses and shaker. The goal is to find a plan of the robot's actions that serves a desired set of drinks.

\subsubsection{Parking}
This domain was created for learning track \textit{IPC-2008}. This domain involves parking cars on a street with N curb locations, and where cars can be double-parked but not triple-parked. The goal is to find a plan to move from one configuration of parked cars to another configuration, by driving cars from one curb location to another. 

\subsubsection{Blocks World}
This domain had origin in the \textit{IPC-2000}. This domain consists of a set of blocks, a table and a robot hand. The blocks can be on top of other blocks or on the table; a block that has nothing on it is clear; and the robot hand can hold one block or be empty. The goal is to find a plan to move from one configuration of blocks to another. The problem probBLOCKS-4-0 is going to be our instance that will be used to explain our results in this Project. We chosen that because is a simple problem that can be resolved in less than few seconds and also contains the characteristics of duplicate nodes.

\section{Problem Formulation}
Let the \textit{underlying search tree} (\textit{UST}) be the full brute-force tree created from a connected, undirected and implicitly defined \textit{underlying search graph} (\textit{USG}) describing a state space. Some search algoritms expand a subtree of the \textit{UST} while searching for a solution (e.g., a portion of the \textit{UST} might not be expanded due to heuristic guidence.); we call this subtree the \textit{expanded search tree} (\textit{UST}). In this paper we want to estimate the size of the subgraph expanded by an algorithm searching the (\textit{USG}); we call this subgraph the \textit{expanded search graph} (\textit{ESG})

\begin{mydef}
\textit{(Canonical and Duplicate Nodes). A node n in the UST is called a duplicate node if there exists another node n' in the UST such that state(s) $=$ state(n') and n' $<$  n. A node that is not duplicate is called a canonical node.}
\end{mydef}

We define three graphs: The first one is the graph generated by dijkstra algorithm (\textit{subtree of the UST}), which contains the tree without duplicate nodes. The second is the Brute Force Search tree generated by the Stratified Sampling algorithm which generate duplicate nodes and the last one is the tree generated by A* using a consistent heuristic which is also the tree we want to predict.\newline

Formally, Let \textit{G = (N, E)} be a  graph representing an expanded search graph (ESG) of  each graph explained above where \textit{N} is its set of states and  for each:
\[  {n \in  N / op(n) =  {op_i|(n, n_i)}  \in E }  \]

 is its set of operators. We use the words edges and operators interchangeably. The prediction is to estimate the size of \textit{N} without fully generate the ESG from A*.

\section{Predicting the size of the Search Graph}
We now explain how we pretend to achieve the prediction of the number of  nodes expanded by A*.\newline

\subsection{Dijkstra Algorithm}
Dijkstra Algorithm is an uninformed search algorithm. In our cost function the value of \textbf{\textit{h}} is \textbf{\textit{0}} for all the nodes of the search tree and the value \textbf{\textit{g}} increase according to the level of the tree. It means that there is no guide to search the goal, in each iteratation the nodes from a level expand blindly according to the value of the level  up to find the goal node. We collect the total number of nodes expanded by level without prunning. So, for our exemple in the level \textbf{\textit{0}} we have one node which is initial state, in the level \textbf{\textit{1}} we have \textbf{\textit{4}} nodes, level \textbf{\textit{2}} contains \textbf{\textit{12}} nodes, etc. This information is collected in the planner executing A* using the dijkstra heuristic which assign the value zero for all the nodes in the search tree. 

\begin{figure}[htb]
\begin{center}
\includegraphics[width=12cm]{./image/dijkstra}
\caption{The par value (\textbf{\textit{h}}, \textbf{\textit{g}}) represent the heuristic value \textbf{\textit{h}} and the cost to go from the start state to the current node \textbf{\textit{g}}. The tree generated by Dijkstra Search Algorithm to the problem of blocks worlds probBLOCKS-4-0.}
\label{fig:dijkstraTreeInconsistent}
\end{center}
\end{figure}

In the Figure~\ref{fig:dijkstraTreeInconsistent} are showed the total number of nodes expanded by level and also how the tree is distributed.\newline

We also, present the Table~\ref{table:tabledijstraExpandedByLevel} which explain the distribution of nodes expanded by level during dijkstra search algorithm.\newline

\footnotesize  % Switch from 12pt to 11pt; otherwise, table won't fit
\setlength\LTleft{100pt}            % default: \fill
\setlength\LTright{200pt}           % default: \fill
\begin{longtable}{@{\extracolsep{\fill}} lll @{}} %\begin{tabular}{|c|c|}
\hline
\textit{Level} && \#Nodes expanded \\ \hline
0       && 1                \\ \hline
1       && 4                \\ \hline
2       && 12               \\ \hline
3       && 24               \\ \hline
4       && 36               \\ \hline
5       && 24               \\ \hline
6       && 24               \\ \hline
\caption{Number of nodes expanded by \textit{Level} using Dijkstra algorithm.}
\label{table:tabledijstraExpandedByLevel}
\end{longtable}

\subsection{\textit{f-value} Distribution using Dijkstra and collecting \textit{f-value} with consistent heuristic - Assumption }
In order to keep the data consistent and validate the process of the prediction we make assumptions about what we expect to obtain, for example we make comparison between \textit{f-value} distributions generated by two algorithms with the objective to see which algorithm is superior and fits fits well to our approach.\newline

\begin{figure}[htb]
\begin{center}
\includegraphics[width=12cm]{./image/dijkstraCollecting}
\caption{The tree generated by Dijkstra Search Algorithm and the solution route expressed by bold solid line. }
\label{fig:dijkstraTreeConsistentHeuristic}
\end{center}
\end{figure}

Starting with our assumptions we present the distribution of \textit{f-values} in the tree generated by dijkstra, but collecting the \textit{f-values} using a consistent heuristic.

The search tree generated by Dijkstra Search Algorithm expand each node blindly with heuristic value zero. In the Figure \ref{fig:dijkstraTreeConsistentHeuristic} the solid line represents the connection from the parent nodes to their childs nodes. Each node is represented with the par \textbf{\textit{(h, g)}} which represent the heuristic value \textbf{\textit{h}} and the cost to go from the origin node to the current node \textbf{\textit{g}}.\newline 

Dijkstra Algorithm iteratively expand each node in each level up to found the solution. The initial node is given by \textbf{\textit{(0, 0)}} and is located at level \textbf{\textit{0}} and the goal node is \textbf{\textit{(0, 6)}} and is at level \textbf{\textit{6}}. We expanded each node and collected the \textbf{\textit{f-value}} based on a consistent heuristic \textbf{\textit{h}} and the \textbf{\textit{g}} value of the node. The bold solid line represent the solution of using a consistent heuristic in the search. The route solution is given by: \textbf{\textit{(6, 0), (5, 1), (4, 2), (3, 3), (2, 4), (5, 1), (0, 6)}}\newline

We also present the table with the distribution of \textit{f-value} during dijkstra algorithm execution.
\footnotesize  % Switch from 12pt to 11pt; otherwise, table won't fit
\setlength\LTleft{100pt}            % default: \fill
\setlength\LTright{200pt}           % default: \fill
\begin{longtable}{@{\extracolsep{\fill}} llll @{}} %\begin{tabular}{|c|c|c|}
\hline
Level              & \textit{f-value} & quantity \\ \hline
0                  & 6       & 1        \\ \hline
\multirow{2}{*}{1} & 6       & 1        \\ \cline{2-3} 
                   & 8       & 3        \\ \hline
\multirow{3}{*}{2} & 6       & 1        \\ \cline{2-3} 
                   & 8       & 3        \\ \cline{2-3} 
                   & 10      & 8        \\ \hline
\multirow{4}{*}{3} & 6       & 1        \\ \cline{2-3} 
                   & 8       & 2        \\ \cline{2-3} 
                   & 10      & 6        \\ \cline{2-3} 
                   & 12      & 15       \\ \hline
\multirow{4}{*}{4} & 6       & 1        \\ \cline{2-3} 
                   & 12      & 2        \\ \cline{2-3} 
                   & 14      & 6        \\ \cline{2-3} 
                   & 16      & 15       \\ \hline
\multirow{4}{*}{5} & 6       & 1        \\ \cline{2-3} 
                   & 12      & 2        \\ \cline{2-3} 
                   & 14      & 6        \\ \cline{2-3} 
                   & 16      & 15       \\ \hline
\multirow{4}{*}{6} & 6       & 1        \\ \cline{2-3} 
                   & 14      & 2        \\ \cline{2-3} 
                   & 16      & 6        \\ \cline{2-3} 
                   & 18      & 15       \\ \hline
\caption{\textit{f-value} distribution expanding the graph using dijkstra and collecting the \textit{f-value} using a consistent heuristic.}
\label{table:tabledijkstra}
\end{longtable}

The Table \ref{table:tabledijkstra} represent the distribution of \textit{f-value} in the tree search expanded by dijkstra and using a consistent heuristic to collect the \textit{f-value}. The first column represent the level in which the node is located. The second column is the \textit{f-value} and the third column is the quantity, which represent the number of nodes in the level that have the same \textit{f-value}.
\newline

In order to make a prediction we will require the Formula~\eqref{eq:partitionFormula} on page \pageref{eq:partitionFormula}. $P_ex$ is the percentage of nodes expanded by level with \textit{f-value} \textbf{\textit{d}} less than to the threshold.

\footnotesize  % Switch from 12pt to 11pt; otherwise, table won't fit
\setlength\LTleft{100pt}            % default: \fill
\setlength\LTright{200pt}           % default: \fill
\begin{longtable}{@{\extracolsep{\fill}} llll @{}} %\begin{tabular}{|c|c|cl}
\hline
\textit{Level} &\rule{4pt}{0ex} Ni &\rule{0pt}{0ex} $P_ex(s, d = 6, \textit{Level})$ \\ \hline
0     &\rule{4pt}{0ex} 1\rule{16pt}{0ex}  & 1                    \\ \hline
1     &\rule{4pt}{0ex} 4\rule{4pt}{0ex}  & 0.250                \\ \hline
2     &\rule{4pt}{0ex} 12 & 0.083                \\ \hline
3     &\rule{4pt}{0ex} 24 & 0.041                \\ \hline
4     &\rule{4pt}{0ex} 36 & 0.027                \\ \hline
5     &\rule{4pt}{0ex} 24 & 0.041                \\ \hline
6     &\rule{4pt}{0ex} 24 & 0.041                \\ \hline
\caption{\textit{f-value} distribution expanding the graph using dijkstra and collecting the \textit{f-value} using a consistent heuristic.}
\label{table:tabledijkstra}
\end{longtable}

Applying the Formula~\eqref{eq:generalFormula} on page \pageref{eq:generalFormula} we obtain the prediction.

\footnotesize  % Switch from 12pt to 11pt; otherwise, table won't fit
\setlength\LTleft{100pt}            % default: \fill
\setlength\LTright{200pt}           % default: \fill
\begin{longtable}{@{\extracolsep{\fill}} lllll @{}}
\hline
                                &\rule{4pt}{0ex} threshold &\rule{4pt}{0ex} \#nodes expanded &\rule{4pt}{0ex} Prediction \\ \hline
probBLOCKS-4-0                  &\rule{4pt}{0ex} 6         &\rule{4pt}{0ex} 7                &\rule{4pt}{0ex} 7          \\ \hline
probBLOCKS-4-1                  &\rule{4pt}{0ex} 10        &\rule{4pt}{0ex} 11               &\rule{4pt}{0ex} 11         \\ \hline
probBLOCKS-4-2                  &\rule{4pt}{0ex} 6         &\rule{4pt}{0ex} 7                &\rule{4pt}{0ex} 7          \\ \hline
probBLOCKS-5-0                  &\rule{4pt}{0ex} 12        &\rule{4pt}{0ex} 16               &\rule{4pt}{0ex} 16         \\ \hline
probBLOCKS-5-1                  &\rule{4pt}{0ex} 10        &\rule{4pt}{0ex} 14               &\rule{4pt}{0ex} 14         \\ \hline
probBLOCKS-5-2                  &\rule{4pt}{0ex} 16        &\rule{4pt}{0ex} 29               &\rule{4pt}{0ex} 29         \\ \hline
probBLOCKS-6-0                  &\rule{4pt}{0ex} 12        &\rule{4pt}{0ex} 13               &\rule{4pt}{0ex} 13         \\ \hline
probBLOCKS-6-1                  &\rule{4pt}{0ex} 10        &\rule{4pt}{0ex} 11               &\rule{4pt}{0ex} 11         \\ \hline
\multirow{3}{*}{probBLOCKS-6-2} &\rule{4pt}{0ex} 18        &\rule{4pt}{0ex} 44               &\rule{4pt}{0ex} 44         \\ \cline{2-4} 
                                &\rule{4pt}{0ex} 19        &\rule{4pt}{0ex} 97               &\rule{4pt}{0ex} 97         \\ \cline{2-4} 
                                &\rule{4pt}{0ex} 20        &\rule{4pt}{0ex} 399              &\rule{4pt}{0ex} 399        \\ \hline
\multirow{4}{*}{probBLOCKS-7-0} &\rule{4pt}{0ex} 17        &\rule{4pt}{0ex} 8                &\rule{4pt}{0ex} 8          \\ \cline{2-4} 
                                &\rule{4pt}{0ex} 18        &\rule{4pt}{0ex} 30               &\rule{4pt}{0ex} 30         \\ \cline{2-4} 
                                &\rule{4pt}{0ex} 19        &\rule{4pt}{0ex} 35               &\rule{4pt}{0ex} 35         \\ \cline{2-4} 
                                &\rule{4pt}{0ex} 20        &\rule{4pt}{0ex} 243              &\rule{4pt}{0ex} 243        \\ \hline
\caption{Prediction of the number of nodes expanded by A* for the first 10 blocks world domain problems.}
\label{table:predictionDijkstra}
\end{longtable}

The Table \ref{table:predictionDijkstra} contains four columns: In the first column we have the blocks world domain problems chosen from fast-downward benchmark. We chosen 10 problems, as well as the number of the problem increases it turns more complex in terms of duplicate and expansions nodes. The second column in the threshold which represent the branches where the nodes are expanded. The third column is the number of nodes expanded by A*  in each threshold, and the last one column is the prediction.\newline

The results in the Table \ref{table:predictionDijkstra} are 100\% accurate, and this predictions are the expected, because we are testing with the same data the two terms of the prediction formula~\eqref{eq:generalFormula}. 

\subsection{\textit{f-value} Distribution using DFS - Assumption }
We have an assumption that if the distribution of the \textit{f-values} of the tree generated by dijkstra have similar structure to the distribution of the \textit{f-values} generated by Depth First Search (DFS). It would mean that the prediction using the number of nodes expanded by level with dijkstra will generate accurate values conforme to the number of nodes by \textit{f-value} generated by A*.\newline

Our assumption about the similarity of the \textit{f-value} distribution of Dijkstra and Depth First Search was tested with the same problem mentioned above obtaining the following behaviors: The \textit{f-value} distribution of Dijkstra was quickly elaborated, three steps were necesary, first the nodes expansions, the collection of the \textit{f-value} of the nodes and then the generation of the \textit{f-value} distribution file. Nevertheless, DFS last approximately less than 5 seconds to expand all the nodes and generate the \textit{f-value} which make sense because it expands just 2581 nodes. For other instances for example the probBLOCKS-6-2 just expand the nodes takes approximitely 6 minutes.\newline

So, try to generate the \textit{f-value} distribution is expensive in computational time because expand nodes the way how DFS expand the nodes is a iterative task that could take much time. So we are going to present the distribution obtained from probBLOCKS-4-0. We found that the nodes \textbf{\textit{(6, 0), (5, 1), (4, 2), (3, 3), (2, 4), (5, 1), (0, 6)}} were expanded in the same way as expanded by dijkstra algorithm, see the Table~\ref{table:tabledijkstra}  on page ~\pageref{table:tabledijkstra}. It means that the distribution of \textit{f-value} equal to 6 in each level generated by DFS is the same as the \textit{f-value} distribution with \textit{f-value} equal to 6 in each level generated by Dijkstra.\newline

According to our expience testing DFS, it could take much time to execute one instance if we set the depth of the search greater than the \textit{f-value} we are going to produce prediction. Then, after analyze the consistent heuristic we are using we set up the depth to be equal to the initial heuristic value. In this way for the problem probBLOCKS-5-0 the depth is going to be \textbf{\textit{6}}.

\footnotesize  % Switch from 12pt to 11pt; otherwise, table won't fit
\setlength\LTleft{100pt}            % default: \fill
\setlength\LTright{200pt}           % default: \fill
\begin{longtable}{@{\extracolsep{\fill}} llll @{}}
\hline

Level               & \textit{f-value} & quantity \\ \hline
0                   & 6       & 1        \\ \hline
\multirow{2}{*}{1}  & 6       & 1        \\ \cline{2-3} 
                    & 8       & 3        \\ \hline
\multirow{3}{*}{2}  & 6       & 1        \\ \cline{2-3} 
                    & 8       & 7        \\ \cline{2-3} 
                    & 10      & 8        \\ \hline
\multirow{4}{*}{3}  & 6       & 1        \\ \cline{2-3} 
                    & 8       & 9        \\ \cline{2-3} 
                    & 10      & 27       \\ \cline{2-3} 
                    & 12      & 15       \\ \hline
\multirow{5}{*}{4}  & 6       & 1        \\ \cline{2-3} 
                    & 8       & 11       \\ \cline{2-3} 
                    & 10      & 59       \\ \cline{2-3} 
                    & 12      & 88       \\ \cline{2-3} 
                    & 14      & 25       \\ \hline
\multirow{6}{*}{5}  & 6       & 1        \\ \cline{2-3} 
                    & 8       & 12       \\ \cline{2-3} 
                    & 10      & 79       \\ \cline{2-3} 
                    & 12      & 239      \\ \cline{2-3} 
                    & 14      & 186      \\ \cline{2-3} 
                    & 16      & 15       \\ \hline
\multirow{7}{*}{6}  & 6       & 1        \\ \cline{2-3} 
                    & 8       & 13       \\ \cline{2-3} 
                    & 10      & 103      \\ \cline{2-3} 
                    & 12      & 507      \\ \cline{2-3} 
                    & 14      & 832      \\ \cline{2-3} 
                    & 16      & 321      \\ \cline{2-3} 
                    & 18      & 15       \\ \hline
\caption{\textit{f-value} distribution expanding the graph using depth first search and collecting the \textit{f-value} using a consistent heuristic and threshold equal to the initial heuristic value.}
\label{table:tabledfs}
\end{longtable}

The Table \ref{table:tabledfs} represent the distribution of \textit{f-value} in the tree search expanded by depth first search using a consistent heuristic and threshold twice the initial heuristic value. Depth first search found seven nodes with \textit{f-value} equal to six in the levels \textbf{\textit{0, 1, 2, 3, 4, 5}}, and  \textbf{\textit{6}} which are the same distribution of \textit{f-value} equal to six  in Table \ref{table:tabledijkstra} on page \pageref{table:tabledijkstra}.\newline


This result encourage to continue with our research, because the distribution of dijkstra is more quickly to obtain and also make prediction over the distribution in comparison with the depth first search tree generated that is more expensive to obtain.\newline


\footnotesize  % Switch from 12pt to 11pt; otherwise, table won't fit
\setlength\LTleft{100pt}            % default: \fill
\setlength\LTright{200pt}           % default: \fill
\begin{longtable}{@{\extracolsep{\fill}} lllll @{}}
\hline
                                &\rule{4pt}{0ex} Threshold &\rule{4pt}{0ex} \#nodes expanded &\rule{4pt}{0ex} Prediction \\ \hline
probBLOCKS-4-0                  &\rule{4pt}{0ex} 6         &\rule{4pt}{0ex} 7                &\rule{4pt}{0ex} 3.54892    \\ \hline
\multirow{5}{*}{probBLOCKS-4-1} &\rule{4pt}{0ex} 4         &\rule{4pt}{0ex} 1                &\rule{4pt}{0ex} 1          \\ \cline{2-4} 
                                &\rule{4pt}{0ex} 5         &\rule{4pt}{0ex} 2                &\rule{4pt}{0ex} 2          \\ \cline{2-4} 
                                &\rule{4pt}{0ex} 6         &\rule{4pt}{0ex} 3                &\rule{4pt}{0ex} 3          \\ \cline{2-4}
                                &\rule{4pt}{0ex} 8         &\rule{4pt}{0ex} 4                &\rule{4pt}{0ex} 6          \\ \cline{2-4}
                                &\rule{4pt}{0ex} 10        &\rule{4pt}{0ex} 14               &\rule{4pt}{0ex} 6          \\ \hline
probBLOCKS-4-2                  &\rule{4pt}{0ex} 6         &\rule{4pt}{0ex} 10               &\rule{4pt}{0ex} 5.12235    \\ \hline
\multirow{6}{*}{probBLOCKS-5-0} &\rule{4pt}{0ex} 6         &\rule{4pt}{0ex} 3                &\rule{4pt}{0ex} 74.6       \\ \cline{2-4}
							   &\rule{4pt}{0ex} 7         &\rule{4pt}{0ex} 4                &\rule{4pt}{0ex} 75.6        \\ \cline{2-4}
							   &\rule{4pt}{0ex} 8         &\rule{4pt}{0ex} 8                &\rule{4pt}{0ex} 80.2        \\ \cline{2-4}
							   &\rule{4pt}{0ex} 10        &\rule{4pt}{0ex} 12               &\rule{4pt}{0ex} 94.2254     \\ \cline{2-4}
							   &\rule{4pt}{0ex} 11        &\rule{4pt}{0ex} 13               &\rule{4pt}{0ex} 97.6516     \\ \cline{2-4}
							   &\rule{4pt}{0ex} 12        &\rule{4pt}{0ex} 30               &\rule{4pt}{0ex} 730         \\ \hline
\multirow{4}{*}{probBLOCKS-5-1} &\rule{4pt}{0ex} 6         &\rule{4pt}{0ex} 3                &\rule{4pt}{0ex} 2.7         \\ \cline{2-4}
							   &\rule{4pt}{0ex} 7         &\rule{4pt}{0ex} 5                &\rule{4pt}{0ex} 4.10741     \\ \cline{2-4}
							   &\rule{4pt}{0ex} 8         &\rule{4pt}{0ex} 11               &\rule{4pt}{0ex} 13.598      \\ \cline{2-4}
							   &\rule{4pt}{0ex} 10        &\rule{4pt}{0ex} 27               &\rule{4pt}{0ex} 60.2934     \\ \hline
\multirow{6}{*}{probBLOCKS-5-2} &\rule{4pt}{0ex} 8         &\rule{4pt}{0ex} 2                &\rule{4pt}{0ex} 36          \\ \cline{2-4}
                                &\rule{4pt}{0ex} 10        &\rule{4pt}{0ex} 4                &\rule{4pt}{0ex} 38          \\ \cline{2-4}
                                &\rule{4pt}{0ex} 12        &\rule{4pt}{0ex} 11               &\rule{4pt}{0ex} 43.3429     \\ \cline{2-4}
                                &\rule{4pt}{0ex} 14        &\rule{4pt}{0ex} 29               &\rule{4pt}{0ex} 64.7618     \\ \cline{2-4}
                                &\rule{4pt}{0ex} 15        &\rule{4pt}{0ex} 36               &\rule{4pt}{0ex} 65.6421     \\ \cline{2-4}
                                &\rule{4pt}{0ex} 16        &\rule{4pt}{0ex} 119              &\rule{4pt}{0ex} 866         \\ \hline
\multirow{3}{*}{probBLOCKS-6-0} &\rule{4pt}{0ex} 10        &\rule{4pt}{0ex} 6                &\rule{4pt}{0ex} 3.65083    \\ \cline{2-4}
							    &\rule{4pt}{0ex} 11       &\rule{4pt}{0ex} 7                &\rule{4pt}{0ex} 4.65083    \\ \cline{2-4}
								&\rule{4pt}{0ex} 12       &\rule{4pt}{0ex} 33               &\rule{4pt}{0ex} 19.3541    \\ \hline
probBLOCKS-6-1                  &\rule{4pt}{0ex} 10        &\rule{4pt}{0ex} 21               &\rule{4pt}{0ex} 17.1216    \\ \hline
\multirow{10}{*}{probBLOCKS-6-2}&\rule{4pt}{0ex} 10        &\rule{4pt}{0ex} 2               &\rule{4pt}{0ex} 187          \\ \cline{2-4} 
                                &\rule{4pt}{0ex} 12        &\rule{4pt}{0ex} 3               &\rule{4pt}{0ex} 1877.667     \\ \cline{2-4}
                                &\rule{4pt}{0ex} 13        &\rule{4pt}{0ex} 4               &\rule{4pt}{0ex} 189          \\ \cline{2-4} 
                                &\rule{4pt}{0ex} 14        &\rule{4pt}{0ex} 10              &\rule{4pt}{0ex} 193.056      \\ \cline{2-4} 
                                &\rule{4pt}{0ex} 15        &\rule{4pt}{0ex} 11              &\rule{4pt}{0ex} 194.343      \\ \cline{2-4} 
                                &\rule{4pt}{0ex} 16        &\rule{4pt}{0ex} 51              &\rule{4pt}{0ex} 212.526      \\ \cline{2-4} 
                                &\rule{4pt}{0ex} 17        &\rule{4pt}{0ex} 59              &\rule{4pt}{0ex} 217.676      \\ \cline{2-4} 
                                &\rule{4pt}{0ex} 18        &\rule{4pt}{0ex} 239             &\rule{4pt}{0ex} 432.239      \\ \cline{2-4} 
                                &\rule{4pt}{0ex} 19        &\rule{4pt}{0ex} 263             &\rule{4pt}{0ex} 481.725      \\ \cline{2-4} 
                                &\rule{4pt}{0ex} 20        &\rule{4pt}{0ex} 762             &\rule{4pt}{0ex} 7057         \\ \hline                                
\multirow{8}{*}{probBLOCKS-7-0} &\rule{4pt}{0ex} 12        &\rule{4pt}{0ex} 1                &\rule{4pt}{0ex} 1           \\ \cline{2-4}
                                &\rule{4pt}{0ex} 13        &\rule{4pt}{0ex} 2                &\rule{4pt}{0ex} 2           \\ \cline{2-4}
                                &\rule{4pt}{0ex} 14        &\rule{4pt}{0ex} 4                &\rule{4pt}{0ex} 3.33333     \\ \cline{2-4}
                                &\rule{4pt}{0ex} 16        &\rule{4pt}{0ex} 8                &\rule{4pt}{0ex} 7.61538     \\ \cline{2-4}
                                &\rule{4pt}{0ex} 17        &\rule{4pt}{0ex} 9                &\rule{4pt}{0ex} 9           \\ \cline{2-4} 
                                &\rule{4pt}{0ex} 18        &\rule{4pt}{0ex} 43               &\rule{4pt}{0ex} 24.5351     \\ \cline{2-4} 
                                &\rule{4pt}{0ex} 19        &\rule{4pt}{0ex} 46               &\rule{4pt}{0ex} 28.3212     \\ \cline{2-4} 
                                &\rule{4pt}{0ex} 20        &\rule{4pt}{0ex} 236              &\rule{4pt}{0ex} 133.908     \\ \hline                              
\caption{Prediction of the number of nodes expanded by A* using the number of nodes by level from dijkstra and the \textit{f-value} distribution obtained from DFS.}
\label{table:predictionDFS}
\end{longtable}

The Table \ref{table:predictionDFS} displays the results of apply the Formula~\eqref{eq:generalFormula} using two terms: The first one, the number of nodes expanded by dijkstra, and the other is the \textit{f-value} distribution obtained from DFS. For the first row the terms are being obtained from the Table~\ref{table:tabledijstraExpandedByLevel} and the Table~\ref{table:tabledfs}.



\subsection{Stratification Sampling}
\cite{knuth1975Estimating} did experiments to predict the number of nodes expanded by IDA* under the assumption that all branches contained the same structure.  He realized that the method was not effective when the branches were unbalanced. \cite{chen1992heuristic} addressed the problem with stratification of the search tree through a \textit{type system}. He assumed that nodes of the same type at a level of the search tree would generate subtrees of the same size. Then, only one node of each type \textbf{SS} estimates the size of the tree.

The distribution the \textit{f-values} is going to be obtained from the Brute Force Search, in this case the \textbf{SS}.

\subsubsection{Type System}
We require a property that allow us to represent each node in the state space search and this property can be defined using any information about the node. For example we could use a \textit{type system} which counts how many children a node generates, or how many children the parent of the current node generates or the parent of the parent of the current node, the \textit{f-value}, the \textbf{\textit{g}} value or the heuristic value \textbf{\textit{h}} of the node, etc. \newline 
This property is used in \textbf{SS} to distinguish the nodes or stratified the state space. In our research we present a \textit{type system} that use the heuristic value of the node \textbf{\textit{h}} and the level in which the node is located \textbf{\textit{L}}. We use the following \textit{type system}:

\[  {T(s) = (h(s), L(s))}  \]

Two nodes at a level of the search tree will have the same \textit{type system T} and randomly one of them will be chosen because the asumption is that both generate the same subtree. Then, the subtree not chosen is removed from the tree and the number of nodes in the chosen node is upated withh the sum of nodes of the subtree of the current node and the nodes of the subtree removed.  



\footnotesize  % Switch from 12pt to 11pt; otherwise, table won't fit
\setlength\LTleft{100pt}            % default: \fill
\setlength\LTright{200pt}           % default: \fill
\begin{longtable}{@{\extracolsep{\fill}} lllll @{}}
\hline
                                &\rule{4pt}{0ex} Threshold &\rule{4pt}{0ex} \#nodes expanded &\rule{4pt}{0ex} Prediction \\ \hline
probBLOCKS-4-0                  &\rule{4pt}{0ex} 6         &\rule{4pt}{0ex} 7                &\rule{4pt}{0ex} 3.61117    \\ \hline
\multirow{5}{*}{probBLOCKS-4-1} &\rule{4pt}{0ex} 4         &\rule{4pt}{0ex} 1                &\rule{4pt}{0ex} 1          \\ \cline{2-4} 
                                &\rule{4pt}{0ex} 5         &\rule{4pt}{0ex} 2                &\rule{4pt}{0ex} 2          \\ \cline{2-4} 
                                &\rule{4pt}{0ex} 6         &\rule{4pt}{0ex} 3                &\rule{4pt}{0ex} 3          \\ \cline{2-4}
                                &\rule{4pt}{0ex} 8         &\rule{4pt}{0ex} 4                &\rule{4pt}{0ex} 6          \\ \cline{2-4}
                                &\rule{4pt}{0ex} 10        &\rule{4pt}{0ex} 14               &\rule{4pt}{0ex} 6          \\ \hline
probBLOCKS-4-2                  &\rule{4pt}{0ex} 6         &\rule{4pt}{0ex} 10                &\rule{4pt}{0ex} 5.12235    \\ \hline
\multirow{6}{*}{probBLOCKS-5-0} &\rule{4pt}{0ex} 6         &\rule{4pt}{0ex} 3                &\rule{4pt}{0ex} 2.6         \\ \cline{2-4}
							   &\rule{4pt}{0ex} 7         &\rule{4pt}{0ex} 4                &\rule{4pt}{0ex} 3.6         \\ \cline{2-4}
							   &\rule{4pt}{0ex} 8         &\rule{4pt}{0ex} 8                &\rule{4pt}{0ex} 8.2         \\ \cline{2-4}
							   &\rule{4pt}{0ex} 10        &\rule{4pt}{0ex} 12               &\rule{4pt}{0ex} 21.8861     \\ \cline{2-4}
							   &\rule{4pt}{0ex} 11        &\rule{4pt}{0ex} 13               &\rule{4pt}{0ex} 25.3694     \\ \cline{2-4}
							   &\rule{4pt}{0ex} 12        &\rule{4pt}{0ex} 30               &\rule{4pt}{0ex} 459         \\ \hline
\multirow{4}{*}{probBLOCKS-5-1} &\rule{4pt}{0ex} 6         &\rule{4pt}{0ex} 3                &\rule{4pt}{0ex} 2.7         \\ \cline{2-4}
							   &\rule{4pt}{0ex} 7         &\rule{4pt}{0ex} 5                &\rule{4pt}{0ex} 4.14     \\ \cline{2-4}
							   &\rule{4pt}{0ex} 8         &\rule{4pt}{0ex} 11               &\rule{4pt}{0ex} 13.8467      \\ \cline{2-4}
							   &\rule{4pt}{0ex} 10        &\rule{4pt}{0ex} 27               &\rule{4pt}{0ex} 59.4738     \\ \hline
\multirow{6}{*}{probBLOCKS-5-2} &\rule{4pt}{0ex} 8         &\rule{4pt}{0ex} 2                &\rule{4pt}{0ex} 2          \\ \cline{2-4}
                                &\rule{4pt}{0ex} 10        &\rule{4pt}{0ex} 4                &\rule{4pt}{0ex} 4          \\ \cline{2-4}
                                &\rule{4pt}{0ex} 12        &\rule{4pt}{0ex} 11               &\rule{4pt}{0ex} 9.36077     \\ \cline{2-4}
                                &\rule{4pt}{0ex} 14        &\rule{4pt}{0ex} 29               &\rule{4pt}{0ex} 32.0507     \\ \cline{2-4}
                                &\rule{4pt}{0ex} 15        &\rule{4pt}{0ex} 36               &\rule{4pt}{0ex} 32.7296     \\ \cline{2-4}
                                &\rule{4pt}{0ex} 16        &\rule{4pt}{0ex} 119              &\rule{4pt}{0ex} 730         \\ \hline
\multirow{3}{*}{probBLOCKS-6-0} &\rule{4pt}{0ex} 10        &\rule{4pt}{0ex} 6                &\rule{4pt}{0ex} 3.64713   \\ \cline{2-4}
							    &\rule{4pt}{0ex} 11       &\rule{4pt}{0ex} 7                &\rule{4pt}{0ex} 4.64713   \\ \cline{2-4}
								&\rule{4pt}{0ex} 12       &\rule{4pt}{0ex} 33               &\rule{4pt}{0ex} 19.1604    \\ \hline
probBLOCKS-6-1                  &\rule{4pt}{0ex} 10        &\rule{4pt}{0ex} 21               &\rule{4pt}{0ex} 17.311    \\ \hline
\multirow{10}{*}{probBLOCKS-6-2}&\rule{4pt}{0ex} 10        &\rule{4pt}{0ex} 2               &\rule{4pt}{0ex} 2          \\ \cline{2-4} 
                                &\rule{4pt}{0ex} 12        &\rule{4pt}{0ex} 3               &\rule{4pt}{0ex} 4     \\ \cline{2-4}
                                &\rule{4pt}{0ex} 13        &\rule{4pt}{0ex} 4               &\rule{4pt}{0ex} 4          \\ \cline{2-4} 
                                &\rule{4pt}{0ex} 14        &\rule{4pt}{0ex} 10              &\rule{4pt}{0ex} 8.04615      \\ \cline{2-4} 
                                &\rule{4pt}{0ex} 15        &\rule{4pt}{0ex} 11              &\rule{4pt}{0ex} 9.34474      \\ \cline{2-4} 
                                &\rule{4pt}{0ex} 16        &\rule{4pt}{0ex} 51              &\rule{4pt}{0ex} 25.9412      \\ \cline{2-4} 
                                &\rule{4pt}{0ex} 17        &\rule{4pt}{0ex} 59              &\rule{4pt}{0ex} 30.6944      \\ \cline{2-4} 
                                &\rule{4pt}{0ex} 18        &\rule{4pt}{0ex} 239             &\rule{4pt}{0ex} 239.922      \\ \cline{2-4} 
                                &\rule{4pt}{0ex} 19        &\rule{4pt}{0ex} 263             &\rule{4pt}{0ex} 281.115      \\ \cline{2-4} 
                                &\rule{4pt}{0ex} 20        &\rule{4pt}{0ex} 762             &\rule{4pt}{0ex} 6317         \\ \hline                                
\multirow{8}{*}{probBLOCKS-7-0} &\rule{4pt}{0ex} 12        &\rule{4pt}{0ex} 1                &\rule{4pt}{0ex} 1           \\ \cline{2-4}
                                &\rule{4pt}{0ex} 13        &\rule{4pt}{0ex} 2                &\rule{4pt}{0ex} 2           \\ \cline{2-4}
                                &\rule{4pt}{0ex} 14        &\rule{4pt}{0ex} 4                &\rule{4pt}{0ex} 3     \\ \cline{2-4}
                                &\rule{4pt}{0ex} 16        &\rule{4pt}{0ex} 8                &\rule{4pt}{0ex} 7.63636     \\ \cline{2-4}
                                &\rule{4pt}{0ex} 17        &\rule{4pt}{0ex} 9                &\rule{4pt}{0ex} 9           \\ \cline{2-4} 
                                &\rule{4pt}{0ex} 18        &\rule{4pt}{0ex} 43               &\rule{4pt}{0ex} 23.8428     \\ \cline{2-4} 
                                &\rule{4pt}{0ex} 19        &\rule{4pt}{0ex} 46               &\rule{4pt}{0ex} 27.6623     \\ \cline{2-4} 
                                &\rule{4pt}{0ex} 20        &\rule{4pt}{0ex} 236              &\rule{4pt}{0ex} 196.257     \\ \hline                              
\caption{Prediction of the number of nodes expanded by A* using the number of nodes by level from dijkstra and the \textit{f-value} distribution obtained from SS.}
\label{table:predictionDFS}
\end{longtable}

\section{Prediction Formula}
For a given state \textbf{\textit{s}} and A* threshold \textbf{\textit{d}}, \textbf{N(s, d)} is the prediction of the number of nodes that A* will expand if it use \textbf{\textit{s}} as its start state and does a complete search with an A* threshold of \textbf{\textit{d}}.

\begin{equation}
\label{eq:generalFormula}
N(s, d) =  \sum\limits_{i=1}^dN_i (s,d)
\end{equation}

Where  \textit{N(s, d)} is the number of nodes expanded by A* at level \textit{i} when its threshold is \textit{d}.
One way to decompose \textit{$N_i(s, d)$} is as the product of two terms. \citep{zahavi2010predicting} 

\begin{equation}
\label{eq:partitionFormula}
N_i (s, d) =  N_i (s) \times P_ex (s, d, i)
\end{equation}
\newline
Where $N_i$ is the number of nodes in level \textit{i} of \textit{BFS}, the brute-force search tree (\textit{i.e.,} the tree created by dijkstra search without heuristic pruning.) of depth \textbf{\textit{d}} rooted at start state \textbf{\textit{s}}, and $P_ex (s, d, i)$ is the percentage of nodes in level \textit{i} of BFS(\textit{i.e,} the distribution with threshold two times the heuristic of the initial state generated by the Stratified Sampling.) that are expanded by A* when its threshold is \textbf{\textit{d}}.


\section{Prediction the number of nodes expanded by IDA*}
The research also includes the prediction of the number of nodes expanded by the Iterative Deepening astar IDA*.
\newline

We have run IDA* in order to obtain the number of nodes expanded by bound up to find the solution. Then, for each bound obtained by IDA* we ran SS using each bound as a threshold and compared the results. The results are very accurate for all the IPC domains.

\footnotesize  % Switch from 12pt to 11pt; otherwise, table won't fit
\setlength\LTleft{100pt}            % default: \fill
\setlength\LTright{200pt}           % default: \fill
\begin{longtable}{@{\extracolsep{\fill}} llll @{}} %\begin{tabular}{|c|c|cl}
\hline
\textit{Time} &\rule{4pt}{0ex} bound &\rule{0pt}{0ex} exp \\ \hline
0.32s     &\rule{4pt}{0ex} 6\rule{16pt}{0ex}  & 7                    \\ \hline
\caption{Running IDA* for probBLOCKS-4-0.pddl we get the number of nodes expanded by bound.}
\label{table:tableidaastar}
\end{longtable}

The number of nodes expanded by bound are displayed in the table \ref{table:tableidaastar}, the first column is the time spent to find all the nodes in a certain bound. The second column is the bound and the third column is the number of nodes expanded.

\footnotesize  % Switch from 12pt to 11pt; otherwise, table won't fit
\setlength\LTleft{100pt}            % default: \fill
\setlength\LTright{200pt}           % default: \fill
\begin{longtable}{@{\extracolsep{\fill}} llll @{}} %\begin{tabular}{|c|c|cl}
\hline
\textit{bound} &\rule{4pt}{0ex} ida* (exp) &\rule{0pt}{0ex} ss (exp) \\ \hline
6     &\rule{4pt}{0ex} 7\rule{16pt}{0ex}  & 7                    \\ \hline
\caption{Running SS for probBLOCKS-4-0.pddl we get the number of nodes expanded by bound.}
\label{table:tableidass}
\end{longtable}

In the table \ref{table:tableidass} is displayed the comparison of number of nodes expanded by ida* and ss given a certain bound.

\section{Comparison between IDA* and SS}
We say that a prediction system \textit{V} dominates another prediction system \textit{V'} if \textit{V} is able to produce more accurate predictions in less time than \textit{V'}. In our tables of results we highlight the runtime and error  of a prediction system if it dominates its competitor. The results presented in this section experimentally show that SS employing greather number of probes dominates SS employing less number of probes.
\newline

In our experiments, prediction accuracy is measured in terms of the \textit{\textbf{Relative Unsigned Error}}, which is calculated as, 

\begin{align}
\dfrac{\sum_{s \in PI} \dfrac{|Pred(s, d) - R(s, d)|}{R(s, d)}}{|PI|}
\end{align}

Where \textit{PI} is the set of prolem instances, \textit{Pred(s, d)} and \textit{R(s, d)} are the predicted and actual number of nodes expanded by IDA* for start state \textit{s} and cost bound \textit{d}. A perfect score according to this measure is 0.00.
\newline

In this experiment we also aim to show that SS produces accurate predictions when an inconsistent heuristic is employed. We show results for SS using the \textit{\textbf{Th}}, which is the type system based on the heuristics value.
\newline

In the table \ref{table:comparison1} we display the average values for the ida* value, ida* time, ss-error (\textit{\textbf{Relative Unsigned Error}}), ss-time and the number of instances solved in each domain. The number of probes used in this table is 1000.


\footnotesize  % Switch from 12pt to 11pt; otherwise, table won't fit
\setlength\LTleft{100pt}            % default: \fill
\setlength\LTright{200pt}           % default: \fill
\begin{longtable}{@{\extracolsep{\fill}} llllll @{}}
\hline

%\begin{table}[h]
%\centering
%\caption{My caption}
%\label{my-label}
%\begin{tabular}{|l|l|l|l|l|l|}
%\hline
Domain                   & ida*        & ida* time & ss-error    & ss-time  & n  \\ \hline
barman-opt11-strips      & ---         & ---       & ---         & ---      & 0  \\ \hline
blocks                   & 2.09693e+09 & 10018.2   & 0.742198    & 1.36583  & 24 \\ \hline
elevators-opt08-strips   & 9.5651e+07  & 20223.9   & 1.58258     & 858.46   & 4  \\ \hline
elevators-opt11-strips   & 1.54128e+08 & 36205.9   & 2.84576     & 1178.09  & 2  \\ \hline
floortile-opt11-strips   & ---         & ---       & ---         & ---      & 0  \\ \hline
nomystery-opt11-strips   & 3.60878e+08 & 2514.16   & 0.474518    & 1.10286  & 14 \\ \hline
openstacks-opt08-adl     & ---         & ---       & ---         & ---      & 0  \\ \hline
openstacks-opt08-strips  & 2.0436e+06  & 15633     & 0.293452    & 772.126  & 7  \\ \hline
openstacks-opt11-strips  & 4.16623e+06 & 35539     & 0.44249     & 1678.33  & 3  \\ \hline
parcprinter-opt11-strips & 3065.69     & 337.491   & 0.000228272 & 0.735385 & 13 \\ \hline
parking-opt11-strips     & 3.06957e+08 & 7448.66   & 0.13508     & 20.748   & 5  \\ \hline
pegsol-opt11-strips      & 54945.3     & 1181.48   & 0.326574    & 39.435   & 20 \\ \hline
scanalyzer-opt11-strips  & 1.07152e+09 & 8163.32   & 0.342058    & 8.32571  & 7  \\ \hline
sokoban-opt08-strips     & 280429      & 9.52462   & 0.00357773  & 3.65385  & 13 \\ \hline
sokoban-opt11-strips     & 404338      & 13.4578   & 0.00236228  & 4.86889  & 9  \\ \hline
tidybot-opt11-strips     & 1.13122e+06 & 355.4     & 0.0955455   & 84.1267  & 3  \\ \hline
transport-opt08-strips   & 15          & 0.12      & 0           & 0.04     & 1  \\ \hline
transport-opt11-strips   & ---         & ---       & ---         & ---      & 0  \\ \hline
visitall-opt11-strips    & 7.60463e+06 & 31.6927   & 0.0745269   & 0.84     & 11 \\ \hline
woodworking-opt08-strips & 9.37402e+07 & 6484.69   & 0.0924333   & 27.8425  & 8  \\ \hline
woodworking-opt11-strips & 2.49863e+08 & 16954.9   & 0.247914    & 59.68    & 3  \\ \hline
%\end{tabular}
%\end{table}
\caption{Using ipdb heuristic and employing 1000 probes for SS algorithm}
\label{table:comparison1}
\end{longtable}

In the table \ref{table:comparison2} we display the average values for ss-error obtained after obtain the table \ref{table:comparison1} using 1, 10, 100, 1000 and 5000 probes.

\footnotesize  % Switch from 12pt to 11pt; otherwise, table won't fit
\setlength\LTleft{100pt}            % default: \fill
\setlength\LTright{200pt}           % default: \fill
\begin{longtable}{@{\extracolsep{\fill}} lllllll @{}}
\hline
%\begin{table}[h]
%\centering
%\caption{My caption}
%\label{my-label}
%\begin{tabular}{|l|l|l|l|l|l|l|}
%\hline
                         & \multicolumn{5}{c}{Probes}                          &                \\ \hline
                         & 1        & 10       & 100      & 1000     & 5000     &                \\ \hline
Domain                   & ss-error & ss-error & ss-error & ss-error & ss-error & ida*           \\ \hline
barman-opt11-strips      & ---      & ---      & ---      & ---      & ---      & ---            \\ \hline
blocks                   & 4.789    & 3.543    & 1.678    & 0.742    & 0.485    & 2096930000.000 \\ \hline
elevators-opt08-strips   & 8.909    & 5.724    & 4.030    & 1.583    & 1.183    & 95651000.000   \\ \hline
elevators-opt11-strips   & 13.495   & 9.043    & 3.898    & 2.846    & 1.771    & 154128000.000  \\ \hline
floortile-opt11-strips   & ---      & ---      & ---      & ---      & ---      & ---            \\ \hline
nomystery-opt11-strips   & 1429.530 & 1.759    & 1.065    & 0.475    & 0.167    & 360878000.000  \\ \hline
openstacks-opt08-adl     & ---      & ---      & ---      & ---      & ---      & ---            \\ \hline
openstacks-opt08-strips  & 0.376    & 0.301    & 0.279    & 0.293    & 0.290    & 2043600.000    \\ \hline
openstacks-opt11-strips  & 0.316    & 0.449    & 0.457    & 0.442    & 0.437    & 4166230.000    \\ \hline
parcprinter-opt11-strips & 0.040    & 0.011    & 0.006    & 0.000    & 0.077    & 3065.690       \\ \hline
parking-opt11-strips     & 3.584    & 1.488    & 0.399    & 0.135    & 0.288    & 306957000.000  \\ \hline
pegsol-opt11-strips      & 1.515    & 0.662    & 0.342    & 0.327    & 0.376    & 54945.300      \\ \hline
scanalyzer-opt11-strips  & 1.353    & 49.072   & 0.167    & 0.342    & 0.036    & 1071520000.000 \\ \hline
sokoban-opt08-strips     & 0.112    & 0.026    & 0.005    & 0.004    & 0.003    & 280429.000     \\ \hline
sokoban-opt11-strips     & 0.073    & 0.001    & 0.004    & 0.002    & 0.001    & 404338.000     \\ \hline
tidybot-opt11-strips     & 1.759    & 0.637    & 0.416    & 0.096    & 0.036    & 1131220.000    \\ \hline
transport-opt08-strips   & 0.000    & 0.000    & 0.000    & 0.000    & 0.000    & 15.000         \\ \hline
transport-opt11-strips   & ---      & ---      & ---      & ---      & ---      & ---            \\ \hline
visitall-opt11-strips    & 1.136    & 0.784    & 0.234    & 0.075    & 0.055    & 7604630.000    \\ \hline
woodworking-opt08-strips & 2.541    & 0.950    & 0.384    & 0.092    & 0.054    & 93740200.000   \\ \hline
woodworking-opt11-strips & 3.093    & 1.808    & 0.354    & 0.248    & 0.116    & 249863000.000  \\ \hline
%\end{tabular}
%\end{table}
\caption{Using ipdb heuristics and employing 1, 10, 100, 1000 and 5000 probes for SS algorithm}
\label{table:comparison2}
\end{longtable}

According to the results of the table \ref{table:comparison2} while the number of probes increases then the ss-error (\textit{\textbf{Relative Unsigned Error}}) is very close to 0.00 (zero) which is the the perfect score.

\section{Conclusions}
In this Project we presented approaches to predict the number of nodes expanded by A* and  IDA*, we take as assumption that the expansion of the number of nodes by level that dijkstra produces can be usefull to our prediction because according to the behavior of the distribution of nodes that dijkstra give us we can extend it using linear regression. Another information we get from the \textit{f-value} distribution that Stratified Sampling give us when search the solution of the problem. Joining both informations we can use a mathematical formula of prediction.\newline

Our approach of predicting the number of nodes expanded by A* produce good results for unit cost domains...  
\newpage
%Imports the bibliography file "references.bib"
\bibliography{references}
%\bibliographystyle{references}

\end{document}

